

\documentclass[a4paper,12pt]{book}

\usepackage{graphicx}
\usepackage{amssymb}
\usepackage{amsmath}
\usepackage{amsfonts}

\usepackage{placeins}


\usepackage[english]{babel}
\usepackage{booktabs}
\usepackage{stfloats}
\usepackage[T1]{fontenc}

\usepackage[titletoc,title]{appendix}

\usepackage{float}
\restylefloat{table}

\usepackage{longtable}

\usepackage{xcolor,colortbl}

\usepackage{mathpazo} % Palatino
\usepackage{avant}    % Avant Garde

\usepackage[margin=2.6cm, bottom=3.cm, top=3.cm, portrait]{geometry}\usepackage{caption}\usepackage[a4paper]{hyperref}\usepackage{adjustbox}\usepackage{enumitem}

 \usepackage[stable]{footmisc}

\renewcommand{\vec}[1]{\mathbf{#1}}

\setlength\parindent{0pt}

\newcommand{\HRule}[1]{\hfill \rule{0.2\linewidth}{#1}}

\makeatletter
\newcommand\primitiveinput[1]
{\@@input #1 }
\makeatother

\definecolor{Gray}{gray}{0.90}

\setcounter{tocdepth}{3}
\setcounter{secnumdepth}{3}

%\renewcommand\thesection{\arabic{section}}

%\renewcommand\appendixname{Appendix}
%\renewcommand\appendixpagename{Appendix}

\begin{document}


\thispagestyle{empty} % Remove page numbering on this page

%----------------------------------------------------------------------------------------
%	TITLE SECTION
%----------------------------------------------------------------------------------------

\colorbox{Gray}{
	\parbox[t]{1.0\linewidth}{
		\centering \fontsize{28pt}{40pt}\selectfont % The first argument for fontsize is the font size of the text and the second is the line spacing - you may need to play with these for your particular title
		\vspace*{0.7cm} % Space between the start of the title and the top of the grey box
		
		\hfill Xsuite \\
		\hfill physics manual

%		\hfill Beam Screens\\
		
		\vspace*{0.7cm} % Space between the end of the title and the bottom of the grey box
	}
}

%----------------------------------------------------------------------------------------

%\vfill
%                \hfill {\huge Bunch length 1.0 ns}
                \vfill

%----------------------------------------------------------------------------------------
%	AUTHOR NAME AND INFORMATION SECTION
%----------------------------------------------------------------------------------------

{\centering \large 
\hfill G. Iadarola, R. De Maria,\\
\hfill M. Fitterer, M. Fjellstrom, P. Hermes \\
%\hfill \\
\hfill CERN - Geneva, Switzerland

\HRule{1pt}} % Horizontal line, thickness changed here

%----------------------------------------------------------------------------------------

\clearpage % Whitespace to the end of the page

\newpage

\tableofcontents

\newpage


\renewcommand*{\arraystretch}{1.4}


\chapter{Xtrack}

XTrack is a 6D single particle symplectic tracking code used to compute the
trajectories of individual relativistic charged particles in circular
accelerators. It has been developed based on SixTrack.

The physical models are collected from the main references
\cite{ripken85,barber87,ripken95,heinemann95,barber96,beam_beam,rf_multipoles},
which contain more details of the derivation of the maps.


\section{Notation and reference frame}

The speed, momentum, energy, rest mass, charge of a particle are indicated
by $v$, $P$, $E$, $m$ and $q$, respectively.  These quantities are
related by the following equations:
\begin{align}
  v&=\beta c &
  E^2-P^2c^2&=m^2c^4 &
  E & = \gamma mc^2 &
  Pc & =\beta E
\end{align}
where $\beta$ and $\gamma$ are the relativistic factors.

In a curvilinear reference frame defined by a constant curvature $h_x$ in the
$\hat X, \hat Z$ plane and parameterized by $s$, the
position of the particle at a time $t$ can be written as:
\begin{align}
  \vec Q(t)= \vec r(s) + x \,\hat x(s) + y\, \hat y(s),
\end{align}
and therefore identified by the coordinates $s, x, y, t$ in the reference frame
defined by $\hat x(s)$ and $\hat y(s)$. In particle tracking, $s$ is normally
used as independent parameter and $t$ as a coordinate.

The electromagnetic fields {\bf E} and {\bf B} can be derived in a curvilinear
reference frame from the potentials $V(x,y,s,t)$ and $\mathbf{A}(x,y,s,t)$, where
\begin{align}
\mathbf{A}(x,y,s,t)=A_x(x,y,s,t) \hat x(s) + A_y(x,y,s,t) \hat y(s) + A_s(x,y,s,t) \hat z(s)
\end{align}
and for which:
\begin{align}
  \mathbf{E}  &= -\nabla V - \frac{\partial \mathbf{A}}{\partial t} 
               = -\partial_x V \hat x - \partial_y V \hat y -
  \frac{1}{1+h x}  \partial_s V \hat z - \partial_t \mathbf{A}\\
  \mathbf{B} &= \nabla\times\mathbf{A}  =
  \left(\partial_y A_s - \frac{\partial_s  A_y}{1+h x} \right) \hat x 
  +\left(\frac{\partial_s A_x-\partial_x (1+h x) A_s }{1+h x} \right)\hat y \\
  &+\quad \left(\partial_x A_y - \partial_y A_x \right) \hat z.
\end{align}
In this reference frame the canonical momenta are:
\begin{align}
  P_x&=m \gamma \dot x + q A_x, &
  P_y&=m \gamma \dot y + q A_y, &
  P_s&=m \gamma \dot s (1 + h x)^2 + q (1 + h x) A_s.
\end{align}
and the energy of a particle and the field is
\begin{align}
E=qV + c \sqrt{(mc)^2
              +\frac{(P_s- q A_s(1+hx))^2}{(1+hx)^2}
              +(P_x-q A_x)^2 + (P_x-q A_x)^2}.
\end{align}







\section{Hamiltonian and particle coordinates}

If $s(t)$ is monotonically increasing, it is possible to derive the equations
of motion using $s$ as the independent parameter, $(-t, E)$ as conjugate coordinates and $P_s$ as Hamiltonian.

\begin{align}
  P_s&= (1+h x) \left( 
       \sqrt{ E^2 - (mc^2)^2
           - (P_x - q A_x)^2
           - (P_y - q A_y)^2}
       +q A_s
    \right)
\end{align}

Since in accelerators the orbits of the 
particles are often a perturbation of the reference trajectory followed by a
particle with rest mass $m_0$, charge $q_0$, speed $\beta_0 c$ and momentum
$P_0$, one could use the following derived quantities that usually assume small
values:


\begin{align}
p(x,y) &= \frac{m_0}{m}\frac{P(x,y)}{P_0}   &
\chi &= \frac{q}{q_0}\frac{m_0}{m} &
a(x,y,s) &= \frac{q_0}{P_0}  A(x,y,s) \\
\delta &= \frac{P \frac{m_0}{m} -P_0}{P_0} &
p_t &= \frac{E \frac{m_0}{m} -E_0}{P_0c} &
p_\sigma &= \frac{E \frac{m_0}{m} -E_0}{\beta_0 P_0c}
\\
\zeta &= s \frac{\beta}{\beta_0} - \beta c t &
\tau &= \frac{s}{\beta_0} - ct &
\sigma &= s - \beta_0 ct &
\end{align}

The pairs $(\zeta, \delta)$, $(\tau, p_\tau)$, $(\sigma, p_\sigma)$ are canonical conjugate and can be generated by the following generating functions \footnote{
%$F_2(q, P, t)$,
%$p = \frac{\partial F_2}{\partial q}$, 
%$Q = \frac{\partial F_2}{\partial P}$,
%$K = H + \frac{\partial F_2}{\partial t}$,
$F_2(-t , p_{\rm new}, s)$,
$E/(P_0 c) = \frac{\partial F_2}{\partial (-t)}$, 
$q_{\rm new} = \frac{\partial F_2}{\partial p_{\rm new}}$,
$H_{\rm new} = H + \frac{\partial F_2}{\partial s}$,


}

\begin{align}
F_2&=x p_x + y p_y + \left(\frac{s}{\beta_0}-ct \right) \left(p_\tau + \frac{1}{\beta_0} \right) \\
F_2&=x p_x + y p_y + \left(s-ct \right) \left(\beta_0 p_\sigma + \frac{1}{\beta_0}  \right) \\
F_2&=x p_x + y p_y + \left(-ct \right) \left(\sqrt{(1+\delta)^2+ m^2 c^4} \right) \\
\end{align}



The Hamiltonian is:

\begin{align}
 H_\delta   &= \frac{p_t}{\beta_0} - \frac{m_0}{m}\frac{P_s}{P_0} &
 H_\tau   &= \frac{p_t}{\beta_0} - \frac{m_0}{m}\frac{P_s}{P_0} &
 H_\sigma &= p_\sigma - \frac{m_0}{m}\frac{P_s}{P_0}
\end{align}

\begin{align}
 H_\delta   &= \frac{p_t}{\beta_0} - (1+h x) \left(
      \sqrt{ (1+\delta)^2
           - (p_x - \chi a_x)^2
           - (p_y - \chi a_y)^2}
       + \chi a_s
    \right) \\
 H_\tau   &= \frac{p_t}{\beta_0} - (1+h x) \left(
      \sqrt{ (1+\delta)^2
           - (p_x - \chi a_x)^2
           - (p_y - \chi a_y)^2}
       + \chi a_s
    \right) \\
  H_\sigma &= p_\sigma - (1+h x) \left(
      \sqrt{ (1+\delta)^2
           - (p_x - \chi a_x)^2
           - (p_y - \chi a_y)^2}
       + \chi a_s
    \right),
\end{align}
where 
\begin{align}
\delta &= \frac{P \frac{m_0}{m} -P_0}{P_0} &
\chi &= \frac{q}{q_0}\frac{m_0}{m}.
\end{align}

The following identities are useful to derive the equation of motion.

\begin{align}
\delta&=\sqrt{p_t^2 + 2 p_t/\beta_0 +1} -1 &
\frac{d \delta}{d p_t}= \frac{p_t+1/\beta_0}{1+\delta} = \frac{1}{\beta} \\
\delta&=\sqrt{\beta_0^2 p_\sigma^2 + 2 p_\sigma +1} -1 &
\frac{d \delta}{d p_\sigma}= \frac{p_\sigma\beta_0^2+1}{1+\delta} = \frac{\beta_0}{\beta}
\end{align}

\section{Cavity time, energy errors and acceleration}

A cavity kick depends on:

\begin{equation}
\sin(2 \pi f T + \phi)
\end{equation}

where T is laboratory time.


For the most general case:

\begin{equation}
\sin(2 \pi f T + \phi) = \sin\left(2 \pi f \frac{s-\sigma}{\beta_0 c}  + \phi \right)
\end{equation}

Most codes drop the term $2 \pi f s / (\beta_0 c)$ that is

\begin{equation}
\sin(2 \pi f T + \phi) \to  \sin\left(- 2 \pi f \frac{\sigma}{\beta_0 c} + \phi\right)
\end{equation}

to make sure that a particle that is syncrhonous to the reference trajectory is in phase with the cavity.


\subsection{Implementing energy errors} 

One can define

\begin{equation}
\begin{aligned}
s &= s_0 + n (L_0-L) + n L\\
f_{\rm rev} &= \beta_0 c / L\\ 
f &= h f_{\rm rev}
\end{aligned}
\end{equation}

where $s_0$ is the path length at the cavity turn at 0, $L_0$ is the design circumference, $n$ is the turn number, $h$ is the harmonic number, L is the new path length with an energy error. Indeed one could write $L=L_0(1 +\eta \delta_s$) where $\eta$ is a constant property of the lattice.

Multiple cavities can have their own defined $L$.

Using these definitions, then


\begin{align}
\sin(2 \pi f T + \phi) =
&\sin\left(2 \pi h f_{\rm rev} \frac{s_0 + n(L_0-L) -\sigma}{\beta_0 c}  + \phi\right)\\
=&\sin\left(2 \pi h f_{\rm rev} \frac{n(L_0-L) -\sigma}{\beta_0 c}  + \phi'\right)
\end{align}


where $\phi'=\frac{2\pi h s_0}{L} + \phi$.


In MAD-X twiss and MAD8, indeed the longitudinal coordinates is directly $\sigma'=n(L_0-L) -\sigma$ and the term $n(L_0-L)$ is added smoothly in each thick element. This forces all the cavities to share the same $L$ or $f_{\rm rev}$.

In SixTrack or MAD-X track, one could simply define a turn dependent phase
\begin{equation}
\phi=\phi_0 + 2 \pi h f_{\rm rev} n(L_0-L)
\end{equation}
which is very general or in alternative add a special element that perform at each turn the following transformation:

\begin{equation}
\sigma_{\rm new}=(L_0-L) -\sigma_{\rm old} 
\end{equation}

\subsection{Acceleration}

Accelaration can be achieved by renormalized the relative variables using a new momentum reference. This has the side effect that the fields of the magnets (expressed in normalized strength) follow the energy ramp and that the cavity frequency (if expressed in terms of the harmonic number (NB we should perhaps change this in the Xtrack interface) is updated.

The re-normalization if done once at each turn is:

\begin{align}
p_{x,\rm new} &= p_{x,\rm old} \frac{P_{0,\rm old}}{P_{0,\rm new}} &
p_{y,\rm new} &= p_{y,\rm old} \frac{P_{0,\rm old}}{P_{0,\rm new}} \\
\delta_{\rm new}&= (\delta_{\rm old}+1) \frac{P_{0,\rm old}}{P_{0,\rm new}} -1 &
p_{t,\rm new} &= \frac{p_{t,\rm old} P_{0,\rm old}c + E_{0,\rm old} - E_{0,\rm new}}{P_{0,\rm new}c} \\
\zeta_{\rm new} &= s\beta \left(\frac{1}{\beta_{0,\rm new}} -
\frac{1}{\beta_{0,\rm old}}\right) - \zeta_{\rm old} &
\tau_{\rm new} &= s\left(\frac{1}{\beta_{0,\rm new}} -  \frac{1}{\beta_{0,\rm old}}\right) - \tau_{\rm old} \\
\end{align}


\section{Beam elements}

\subsection{Drift}
A drift is a straight, field-free region ($h(x,y)=0$, $V=0$ and
$\mathbf{A}=0$).  The exact and expanded Hamiltonian for a drift space are
\begin{align}
  H_\tau = \frac{p_t}{\beta_0} - \sqrt{(1+\delta)^2 - p_x^2 - p_y^2} &\approx
  \frac{p_t}{\beta_0} - \delta + \frac{1}{2}\frac{p_x^2+p_y^2}{1+\delta}.
\end{align}
\begin{align}
  H_\sigma = p_\sigma - \sqrt{(1+\delta)^2 - p_x^2 - p_y^2} &\approx
  p_\sigma - \delta + \frac{1}{2}\frac{p_x^2+p_y^2}{1+\delta}.
\end{align}








\subsubsection{Expanded Drift}

The map relative to the expanded Hamiltonian is (eq. 3.49 in \cite{ripken95})
\begin{align}
  x_p &= \frac{p_x}{1+\delta} & 
  y_p &= \frac{p_y}{1+\delta}  \\
  x & \leftarrow x + x_p l &
  y & \leftarrow y + y_p l
\end{align}
\begin{align}
  \tau & \leftarrow \tau +
   \frac{l}{\beta_0} - \frac{l}{\beta} -
    \frac{l}{\beta} \frac{x_p^2+y_p^2}{2}=
    \tau+
    l\left(\frac{\delta}{\beta_0}-\frac{p_t}{1+\delta} - \frac{x_p^2+y_p^2}{2\beta}\right)
\end{align}
\begin{align}
  \sigma & \leftarrow \sigma +
   l - \frac{\beta_0}{\beta} l -
    \frac{\beta_0}{\beta} l \frac{x_p^2+y_p^2}{2}=
    \sigma+
    l\left(1- \frac{\beta_0}{\beta}\left(1 + \frac{x_p^2+y_p^2}{2}\right)\right)
\end{align}

\subsubsection{Exact Drift}

The map relative to the exact Hamiltonian is (eq. 3.49 in \cite{ripken95})
\begin{align}
  p_z&=\sqrt{(1+\delta)^2 - p_x^2 - p_y^2} \\
  \frac{d p_z}{d p_t}&= \frac{p_t+1/\beta_0}{p_z} = \frac{1}{\beta_z}  \\
  \frac{d p_z}{d p_\sigma}&= \frac{\beta_0^2 p_\sigma+1}{p_z} = \frac{\beta_0}{\beta_z}  
\end{align}
\begin{align}
  x & \leftarrow x + \frac{p_x}{p_z} l  &
  y & \leftarrow y + \frac{p_x}{p_z} l
\end{align}
\begin{align}
  \tau & \leftarrow \tau + \frac{l}{\beta_0}
  -\frac{l}{\beta_z}=
  l\left(\frac{1}{\beta_0}-\frac{p_t+1/\beta_0}{p_z}\right)
\end{align}
\begin{align}
  \sigma & \leftarrow \sigma + l
  -l \frac{\beta_0}{\beta_z}=
  l\left(1-\frac{\beta_0^2 p_\sigma+1}{p_z}\right)
\end{align}




\subsubsection{Polar Drift}
It is possible to define a ``polar'' drift that has the effect of rotating the reference frame
\cite{forest99} for instance in the $x$-$z$ plane

\begin{align}
p_x & \leftarrow   p_x \cos \theta + p_z \sin\theta &
p_z & \leftarrow - p_x \sin \theta + p_z \cos\theta \\
z   &= -x \sin \theta & x' &= p_x/p_z &  y' &= p_y/p_z \\
x   & \leftarrow x \cos\theta - x' z  &
y   & \leftarrow y - x' z  & \tau & \leftarrow \tau + z/\beta_z .
\end{align}
where $\theta$ is the angle bringing the new $\hat x$ towards the old $\hat z$.
The map can be also generated by combining a rotation with a $-x
\sin(\theta)$-length drift. In case of an $\hat x$ rotation the role of $x$ and $y$ are interchanged. 


\subsection{Dipole}

In a curvilinear reference system with a constant curvature $h$ in the
horizontal plane a uniform magnetic field can be derived by the vector potential:
\begin{align}
  A_x & = 0, & A_y & = 0, & A_s & = 
  - B_y \left(x-\frac{h x^2}{2 (1+h x)}\right).
\end{align}

With the following normalization $k_0=\frac{q_0}{p} B_y$ is the inverse of the bending 
radius of the reference particle.

The exact and expanded Hamiltonian for a horizontal bending magnet is (eq. 2.12 in
\cite{barber87})
\begin{align}
  H &= \frac{p_t}{\beta_0} 
       - (1+h x)\sqrt{(1+\delta)^2 -p_x^2 - p_y^2}
       + \chi k_0 \left( x + \frac{h x^2}{2} \right)  \\
    &\simeq   \frac{p_t}{\beta_0}
    + \frac{1}{2}\frac{p_x^2+p_y^2}{1+\delta}
  - (1+h x) (1+\delta) + \chi k_0 \left( x + \frac{h x^2}{2} \right)
\end{align}


\subsubsection{Thin dipole}
The map for a thin dipole kick (horizontal or vertical) from the expanded Hamiltonian is 
(eq. 4.12 in \cite{heinemann95}):
\begin{align}
  p_x &\leftarrow p_x + (h_x l - \chi k_0 l)  + h_x l \delta - \chi k_0 l h_x x \\
  p_y &\leftarrow p_y - (h_y l - \chi \hat k_0 l) - h_y l \delta + \chi k_0 l h_y y\\
  \tau &\leftarrow \tau - \frac{h_xx - h_yy}{\beta}  l.
\end{align}

\subsubsection{mad thin dipole}

to be checked
%??? dipr 
\begin{align}
  p_x &\leftarrow p_x + (h_x l - \chi k_0 l)  + h_x l (p_z -1) - \chi k_0 l h_x x \\
  p_y &\leftarrow p_y - (h_y l - \chi \hat k_0 l) - h_y l (p_z-1) + \chi k_0 l h_y y\\
  \tau &\leftarrow \tau - \frac{h_xx - h_yy}{\beta_z}  l.
\end{align}


 %   track(2,jtrk) = track(2,jtrk) - (dbr + dxt(jtrk) - dipr * (ttt - one))
 %    track(4,jtrk) = track(4,jtrk) + (dbi + dyt(jtrk) - dipi * (ttt - one))
 %    track(5,jtrk) = track(5,jtrk) - &
 %         (dipr*track(1,jtrk) - dipi*track(3,jtrk)) *   &
 %         ((one + bet0*track(6,jtrk))/ttt) * bet0i



\subsubsection{Thick dipole}
Defining the following quantities,
\begin{align}
  G_x&= \chi \frac{k_0 h_x}{1+\delta}, & G_y&= \chi \frac{ \hat k_0 h_y}{1+\delta} \\
  C_{x,y}&=\cos(\sqrt{G_{x,y}}L), & S_{x,y}&=\sin(\sqrt{G_{x,y}}L)
\end{align}
the map relative to the expanded Hamiltonian is (eq. 4.11 in \cite{barber87})
\begin{align}
  x   &\leftarrow C_x \cdot x + \frac{S_x}{\sqrt{G_x}}\frac{1}{1+\delta} \cdot p_x + \frac{\delta}{h_x} (1 - C_x) \\
  p_x &\leftarrow -\sqrt{G_x} (1+\delta) \cdot S_x \cdot x + C_x \cdot p_x + \delta \sqrt{1+\delta} \cdot S_x \\
  y   &\leftarrow C_y \cdot y + \frac{S_y}{\sqrt{G_y}}\frac{1}{1+\delta} \cdot p_y + \frac{\delta}{h_y} (1 - C_y) \\
  p_y &\leftarrow -\sqrt{G_y} (1+\delta) \cdot S_y \cdot y + C_y \cdot p_y + \delta \sqrt{1+\delta} \cdot S_y \\
  \sigma &\leftarrow \sigma + L\left(1 - \frac{\beta_0}{\beta}\right) \\
  & \qquad\, -\frac{\beta_0}{\beta} \Bigg[ \frac{h_x S_x}{\sqrt{G_x}} \cdot x + \frac{1-C_x}{h_x} \cdot p_x
  + \frac{h_y S_y}{\sqrt{G_y}} \cdot y + \frac{1-C_y}{h_y} \cdot p_y
  + \delta \left(2L - \frac{S_x}{\sqrt{G_x}} - \frac{S_y}{\sqrt{G_y}} \right) \Bigg] \\
  & \qquad\, - \frac{1}{4}\frac{\beta_0}{\beta} \Bigg[ G_x \left(L-\frac{C_xS_x}{\sqrt{G_x}} \right)
  \left(x - \frac{\delta}{h_x}\right)^2
  + \left(L+\frac{C_xS_x}{\sqrt{G_x}} \right) \frac{p_x^2}{(1+\delta)^2}
  -\left(x-\frac{\delta}{h_x}\right) \frac{2S_x^2}{1+\delta} \cdot p_x \\
  & \qquad\, + G_y \left(L-\frac{C_yS_y}{\sqrt{G_y}} \right)
  \left(y - \frac{\delta}{h_y}\right)^2 + \left(L+\frac{C_yS_y}{\sqrt{G_y}}\right) 
  \frac{p_y^2}{(1+\delta)^2}
  -\left(y-\frac{\delta}{h_y}\right)\frac{2S_y^2}{1+\delta} \cdot p_y \Bigg].
\end{align}

\subsubsection{Dipole Edge effects}
Considering the dipole edge effects from a dipole of length $L$ and bending angle $\theta$, 
the map is
\begin{align*}
    p_x &\to p_x + \frac{1+\delta}{\rho} \tan(\alpha) \cdot x \\
    p_y &\to p_y - \frac{1+\delta}{\rho} \tan(\alpha) \cdot y,
\end{align*}
where the bending radius $\rho$ and $\alpha$ are defined as
\begin{align*}
    \rho^{-1}   &= \frac{h_x}{\sqrt{1+\delta}} &
    \alpha &= \frac{1}{2} \frac{L}{\rho} = \frac{\theta}{2}.
\end{align*}




\subsection{Combined dipole quadrupole}

The following vector potential in curvilinear coordinates
\begin{align}
A_s= -\frac{g}{1+h x}
    \left(\frac{x^2}{2} - \frac{y^2}{2} +
          \frac{h x^3}{3} \right)
\end{align}
produce a field 
\begin{align}
B_x&= g \left(y + \frac{h x y}{1+h x} \right) &
B_y&= g x 
\end{align}

The following vector potential in curvilinear coordinates
\begin{align}
A_s= -\frac{g}{1+h x}
    \left(\frac{x^2}{2} - \frac{y^2}{2} +
          \frac{h x^3}{3} -   \frac{h x y^2}{2} \right)
\end{align}
produce a field 
\begin{align}
B_x&= g y &
B_y&= g \left(x + \frac{h y^2}{2+2 h x} \right) 
\end{align}


\subsection{Thin Multipole}

The effect of a thin multipole can be approximated by the following Hamiltonian

A longitudinally uniform static magnetic field can be described by the following equations
\begin{align}
    B_y+iB_x&=\sum_{n=1}     \frac{B_n+iA_n}{r_0^{n-1}} (x+iy)^{n-1} \\
            &=B_N \sum_{n=N} \frac{b_n+ia_n}{r_0^{n-1}} (x+iy)^{n-1}  .
\end{align}

Usually multipole are expressed as relative to 

A thin multiple idealize the effect of the field by taking the limit of the integration 
length going to zero while keeping constant the integrated strength. The Hamiltonian is:
\begin{align}
  H=- \delta(s) \chi L \Re\left[\sum_{n=0} (k_n + i\hat k_n) (x+iy)^{n+1} \right].
\end{align}
where
\begin{align}
  k_n     &=  n!\frac{q_0}{p_0}  \frac{B_{n+1}}{r_0^n}  &
  \hat k_n&=  n!\frac{q_0}{p_0}  \frac{A_{n+1}}{r_0^n} .
\end{align}


The corresponding map is:
\begin{align}
  p_x &\leftarrow p_x - \chi L\cdot\Re\left[\sum_{n=0} \frac{1}{n!} (k_n + i\hat k_n) (x+iy)^n \right], \\
  p_y &\leftarrow p_y + \chi L\cdot\Im\left[\sum_{n=0} \frac{1}{n!} (k_n + i\hat k_n) (x+iy)^n \right],
\end{align}

In case a curvature $h$ is the vector potential become:

\begin{align}
f(x,y)&=\int B_x(x,y) dy  \\
g(x,y)&=\int \partial x  B_x(x,y) dy \\
a_s(x,y)&=\frac{c_1}{1 + h x} + f(x,y) -
   \frac{\int_1^x (1 + h \xi) (g(\xi,y)+\xi) +h f(x,y)  \, d\xi}{1+ h x}
\end{align}

\begin{align}
\frac{\int_1^x \left(-h \xi \left(g(x,y)\right)-\int \text{bx}^{(1,0)}(\xi,y) \, dy-h \int \text{bx}(\xi,y) \,
   dy-h \xi \text{by}(\xi,y)-\text{by}(\xi,y)\right) \, d\xi}{h x+1}
\end{align}



\subsection{Accelerating Cavity}

The approximated energy gain of a particle passing through an electric field of frequency $f=\frac{k}{2\pi c}$ for which:
\begin{align}
V \sin(\phi - k \tau) = \int_{-l/2}^{l/2} E_s(0,0,t,s)  {\rm \,d}s.
\end{align}

An equivalent vector potential can be derived and normalized as
\begin{align}
A_s& = - \frac{V}{\omega} \cos(\phi - k \tau ) & 
V_n&=  \frac{q_0}{P_0 c} V  & 
\end{align}
from which one can derive the following map
\begin{align}
p_t & \leftarrow p_t + \chi V_n \sin(\phi - k \tau + k \frac{s-s_0}{\beta_0}  ),
\end{align}
where the additional terms in the phase is added in case harmonic number is not exactly integer and the phase is unlocked phase ). The new $\delta$ can be updated from the new $p_t$.



\subsection{RF-Multipole}

The RF-multipole generalizes the interaction of a particle with an electromagnetic field by assuming that

\begin{align}
\Delta E(x,y,\tau) &= q \int_{-L/2}^{L/2} E_z(x,y,t)  {\rm \,d}s \\
\Delta P_x(x,y,\tau) &= q \int_{-L/2}^{L/2} E_x(x,y,t) + \beta c B_y(x,y,t) {\rm \,d}s\\
\Delta P_y(x,y,\tau) &= q \int_{-L/2}^{L/2} E_y(x,y,t) - \beta c B_x(x,y,t) {\rm \,d}s.
\end{align}
are harmonic in $x,y$ and periodic in $\tau$ of frequency $f=\frac{k}{2\pi c}$ such that:

\begin{align}
a_s(x,y,\tau) 
&= \Re \left[ \sum_{n=1}^N
      \left(       k_n \cos(\phi_n -k \tau ) +
            i \hat k_n \cos(\hat \phi_n -k \tau)
      \right)    
      (x+i y )^n
     \right],
\end{align}

The map then follows:
\begin{align}
    \Delta p_x &= -\sum_{n=1}^N \frac{\chi}{n!} \Re\left[ (k_n C_n + i \hat k_n \hat C_n)(x+iy)^{(n-1)}\right], \\
    \Delta p_y &=  \sum_{n=1}^N \frac{\chi}{n!} \Im\left[ (k_n C_n + i \hat k_n \hat C_n)(x+iy)^{(n-1)}\right], \\
    \Delta p_t &= -\chi k \sum_{n=1}^N \Re\left[( k_n S_n + i k_n \hat S_n ) (x+iy)^n\right],
\end{align}
where
\begin{align}
     C_n&=\cos(\phi_n-\omega \Delta t) &
\hat C_n&=\cos(\hat \phi_n-\omega \Delta t) \\
     S_n&=\sin(\phi_n-\omega \Delta t) &
\hat S_n&=\sin(\hat \phi_n-\omega \Delta t) .
\end{align}


\subsection{Solenoid}
The expanded Hamiltonian for a particle in a solenoid is
\begin{align*}
  H &= p_\sigma+\frac{1}{2}\frac{(p_x+R\cdot y)^2+(p_y-R\cdot x)}{1+\delta},
\end{align*}
where $R=\frac{1}{2}\frac{q}{P_0c}\mathbf{B}(0,0,s)$. The map for a solenoid 
of length $L$ in the thin lens approximation with the expanded 
Hamiltonian (eq. 4.35 in \cite{heinemann95})
\begin{align*}
  x   &\to \,\,\,\,\, C\cdot x + S\cdot y \\
  p_x &\to -\theta R\cdot C \cdot x+C\cdot p_x
        -\theta R\cdot S\cdot y+S\cdot p_y \\
  y   &\to -S\cdot x + C\cdot y \\
  p_y &\to \,\,\,\,\, \theta R\cdot S\cdot x - S\cdot p_x 
  - \theta R\cdot C \cdot y + C\cdot p_y \\
  \sigma &\to \sigma - \frac{\beta_0}{\beta}
  \frac{\theta}{1+\delta} \left(\frac{1}{2}
  R (x^2 + y^2) + (p_xy - p_yx)\right)
\end{align*}
where $R\equiv R(s_0)$, $\theta=\frac{R}{1+\delta}$, 
$C=\cos(\theta)$ and $S=\sin(\theta)$.\\[0.5em]
The map for a thick solenoid is (eq. 3.47, 3.48 in \cite{ripken85}) 
\begin{align*}
    x &\to C^2 \cdot x+\frac{1}{R}\cdot S\cdot C\cdot p_x+S\cdot C\cdot y
    + \frac{1}{R}\cdot S^2 \cdot p_y \\
    p_x &\to -R\cdot S\cdot C\cdot x + C^2\cdot p_x
    -R\cdot S^2\cdot y + S\cdot C\cdot p_y \\
    y &\to -S\cdot C\cdot x -\frac{1}{R}\cdot S^2 \cdot p_x
    +C^2\cdot y + \frac{1}{R}\cdot S\cdot C\cdot p_y \\
    p_y &\to R\cdot S^2\cdot x -S\cdot C\cdot p_x
    -R\cdot S\cdot C\cdot y + C^2\cdot p_y \\
    \sigma &\to \sigma - \frac{L}{2} \left[R^2(x^2+y^2)+2R\left(
    \frac{p_x}{1+\delta}y - \frac{p_y}{1+\delta}x\right) 
    + \frac{p_x^2+p_y^2}{(1+\delta)^2} \right]
\end{align*}
where $\theta=R\cdot L$, $C=\cos\theta$ and $S=\sin\theta$.


%\section{Beam-Beam}
%
%The closed expression of an integrated kick experienced by a test particle of charge  $q=Z e$ due to a 2D uncoupled Gaussian charge distribution of total charge $q_b=N_b e$ defined by $\sigma_x, \sigma_y$ and located at $\Delta x, \Delta y$ from the test particle and moving at $v_z=\beta_b c$ from the test particle can be obtained through the use of complex the function \cite{bassetti-erskine}:
%
%\begin{align}
%w(z)=\exp({-z^2})\left(1+\frac{2i}{\sqrt{\pi}}\int_0^z \exp({\xi^2}) d\xi \right)=
%     \exp({-z^2}){\rm erfc}(-iz).
%\end{align}
%
%The kick can then be expressed as
%\begin{equation}
%\Delta p_y+i \Delta p_x= K_b \frac{\sqrt{\pi}}{r}
%\left( {\rm w}\left(\xi_1 \right)
%       -\exp\left( \xi_2^2-\xi_1^2 \right)
%         {\rm w}\left( \xi_2 \right)
%\right)
%\end{equation}
%or, when $\sigma_x=\sigma_y=\sigma$, with
%\begin{equation}
%\Delta p_y+i \Delta p_x= K_b \frac{i \Delta x+ \Delta y}{\Delta_x^2+\Delta y^2}
%\left( 1 - \exp\left(-\frac{\Delta x^2+\Delta y^2}{2\sigma^2}\right) \right)
%\end{equation}
%since $w(z) \leftarrow i / (\sqrt{\pi} z) $ for $z\leftarrow \infty$ with and where
%\begin{align}
%r&=\sqrt{2(\sigma_x^2 - \sigma_y^2)} &
%\xi_1&=\frac {\Delta x}r +i \frac {\Delta y}r &
%\xi_2= \frac{\Delta x\sigma_y}{r\sigma_x}+i \frac{\Delta y \sigma_x}{r\sigma_y}.
%\end{align}
%and where
%\begin{align}
%K_b 
%=\chi \frac{q_0 q_b  (1 -\beta \beta_b)}{2\pi \epsilon_0  P_0 c^2}
%=\frac{2 N_b Z r_0 (1 -\beta \beta_b)}{ \beta_0 \gamma_0} && 
%r_0=\frac{e^2}{4\pi\epsilon_0 m c^2},
%\end{align}
%
%In case the 2D distribution approximates the integrated effect of a longitudinal distribution (e.g. beam beam effect) the kick has to be scaled by a time of flight factor
%\begin{align}
%K_b^{BB}&=\frac{K_b}{\beta - \beta_b}, 
%\end{align}
%while in case the distribution is stationary of length L and charge density $\lambda$ such that $N_b=\lambda L$.
%\begin{align}
%K_b^{SC}&=\frac{K_b}{\beta}. 
%\end{align}
%
%In case the distribution is not longitudinally uniform and the potential of a slice of charge can be written as
%$q_b U(X,Y,Z,t)/(P_0 c)$ where $S=\frac{Z-\beta_0\tau}{2}$ is the location of the interaction point, 
%one can apply the so-called Synchro-Betatron mapping in the ultra-relativistic limit:
%\begin{align}
%x &\leftarrow x- S f_X & 
%y &\leftarrow y- S f_Y   \\
%p_x &\leftarrow p_x+  f_X=x'   &
%p_y &\leftarrow p_y+  f_Y=y' 
%\end{align} 
%\begin{align}
%p_t & \leftarrow p_t  + \frac{1}{2}
%   \left( f_Z 
%        + f_X \frac{p_x +x'}{2} 
%        + f_Y \frac{p_y +y'}{2} \right)
%\end{align} with $f_{X,Y,Z}=- n \partial_{X,Y,Z} U(X=x+S p_x,Y=y+S p_y,Z=Z,t=\tau)$.
%
%In the non-relativistic limit:
%\begin{align}
%f_{X,Y} &= - q_b \frac{1 -\beta \beta_b}{\beta - \beta_b} \partial_{X,Y} U&
%f_Z &= - q_b \frac{1}{\beta - \beta_b} \partial_Z U \\
%S&=\beta_0 \frac{Z/\beta-\tau}{\beta - \beta_b}&
%p_t & \leftarrow p_t  + \frac{\beta_0}{\beta-\beta_b}\left(\ldots\right)
%\end{align}.

\subsection{AC-dipole}
The excitation amplitude of the AC-dipole is denoted by $A$ [Tm], the excitation frequency by $q_d$ [$2\pi$] and the phase of the excitation by $\phi$. The map
presented here is for a purely horizontal dipole, the map for a vertical dipole is obtained by replacing $p_x\to p_y$.

The effect of the AC-dipole is split into four stages. The turn number is denoted by $n$.
\begin{enumerate}
  \item A number of free turns $n_{\text{free}}$, in which the AC-dipole has no effect on the motion.
  \item Ramp-up of the voltage from $0$ to the excitation amplitude $A$ for $n_{\text{ramp-up}}$ turns.
        \begin{align*}
            n' &= \frac{n-n_{\text{free}}}{n_{\text{ramp-up}}} \\
            p_x &\to p_x + n' \cdot \frac{A}{pc} \cdot(1+\delta) \sin\left(2\pi q_d\cdot(n-n_{\text{free}})+\phi\right)
        \end{align*}
  \item Constant excitation amplitude for $n_{\text{flat}}$ turns.
        \begin{align*}
            p_x &\to p_x + \frac{A}{pc}\cdot(1+\delta)\sin\left(2\pi q_d\cdot(n-n_{\text{free}})+\phi\right)
        \end{align*}
  \item Ramp-down of the voltage from the excitation amplitude $A$ to $0$ for $n_{\text{ramp-down}}$ turns.
        \begin{align*}
            n' &= \frac{n-n_{\text{free}}-n_{\text{ramp-up}}-n_{\text{flat}}-n_{\text{ramp-down}}}{n_{\text{ramp-down}}} \\
            p_x &\to p_x + n' \cdot \frac{A}{p} \cdot(1+\delta) \sin\left(2\pi q_d\cdot(n-n_{\text{free}})+\phi\right)
        \end{align*}
\end{enumerate}

\subsection{Wire}
For each part we define $p_z=\sqrt{(1+\delta)^2-x'^2-y'^2}$, using the current values for $x'$ and $y'$.

Step 1. Initial backwards drift of length $L=\frac{embl}{2}$.
\begin{align*}
	x &\to x - L\cdot\frac{x'}{p_z} \\
    y &\to y - L\cdot\frac{y'}{p_z}
\end{align*}

Step 2.
\begin{align*}
	y &\to y - \frac{x\cdot\sin(t_x)}{
    \cos\left(\arctan\left(\frac{x'}{p_z}\right)-t_x\right)}\cdot
    \frac{y'}{\sqrt{(1+\delta)^2-y'^2}} \\
    x &\to x\cdot\left[\cos(t_x)-\sin(t_x)\cdot\tan\left(
    \arctan\left(\frac{x'}{p_z}\right)-t_x\right)\right] \\
    x' &\to \sqrt{(1+\delta)^2-y'^2}\cdot\sin\left(
    \arctan\left(\frac{x'}{p_z}\right)-t_x\right) \\
    x &\to x- \frac{y\cdot\sin(t_y)}{\cos\left(\arctan\left(\frac{y'}{p_z}\right)
    -t_y\right)} \cdot\frac{x'}{\sqrt{(1+\delta)^2-x'^2}} \\
    y &\to y \left[ \cos(t_y) - \sin(t_y)\cdot\tan\left(
    \arctan\left(\frac{y'}{p_z}\right)-t_y\right) \right] \\
    y' &\to \sqrt{(1+\delta)^2-x'^2}\sin\left(\arctan\left(
    \frac{y'}{p_z}\right)-t_y\right)
\end{align*}

Step 3. Drift part of length $L=lin$.
\begin{align*}
    x &\to x + L \cdot \frac{x'}{p_z} \\
    y &\to y + L \cdot \frac{y'}{p_z}
\end{align*}

Step 4. Here $x_i=x-r_x$ and $y=y-r_y$.
\begin{align*}
    x' &\to x' - \frac{\frac{cur\cdot10^{-7}}{chi}\cdot x_i}{x_i^2+y_i^2}
    \left[\sqrt{(lin+l)^2+x_i^2+y_i^2}-\sqrt{(lin-l)^2+x_i^2+y_i^2} \right] \\
    y' &\to y' - \frac{\frac{cur\cdot10^{-7}}{chi}\cdot y_i}{x_i^2+y_i^2}
    \left[\sqrt{(lin+l)^2+x_i^2+y_i^2}-\sqrt{(lin-l)^2+x_i^2+y_i^2} \right]
\end{align*}

Step 5. Drift of length $L=leff-lin$.
\begin{align*}
    x &\to x + L \frac{x'}{p_z} \\
    y &\to y + L \frac{y'}{p_z}
\end{align*}

Step 6.
\begin{align*}
	x &\to x - \frac{y\cdot\sin(-t_y)}{
    \cos\left(\arctan\left(\frac{y'}{p_z}\right)+t_y\right)}\cdot
    \frac{x'}{\sqrt{(1+\delta)^2-x'^2}} \\
    y &\to y\cdot\left[\cos(-t_y)-\sin(-t_y)\cdot\tan\left(
    \arctan\left(\frac{y'}{p_z}\right)+t_y\right)\right] \\
    y' &\to \sqrt{(1+\delta)^2-x'^2}\cdot\sin\left(
    \arctan\left(\frac{y'}{p_z}\right)+t_y\right) \\
    y &\to y- \frac{x\cdot\sin(-t_x)}{\cos\left(\arctan\left(\frac{x'}{p_z}\right)
    +t_x\right)} \cdot\frac{y'}{\sqrt{(1+\delta)^2-y'^2}} \\
    x &\to x \left[ \cos(-t_x) - \sin(-t_x)\cdot\tan\left(
    \arctan\left(\frac{x'}{p_z}\right)+t_x\right) \right] \\
    x' &\to \sqrt{(1+\delta)^2-y'^2}\cdot\sin\left(\arctan\left(
    \frac{x'}{p_z}\right)+t_x\right)
\end{align*}

Step 7. Shift.
\begin{align*}
    x &\to x + embl\cdot \tan(t_x) \\
    y &\to y + embl\cdot\frac{\tan(t_y)}{\cos(t_x)}
\end{align*}

Step 8. Negative drift of length $L=\frac{embl}{2}$.
\begin{align*}
    x &\to x - L\cdot \frac{x'}{p_z} \\
    y &\to y - L\cdot \frac{y'}{p_z}
\end{align*}


\subsection{Misalignment}

Misalignments of elements affects the coordinates at the entrance of an
element as follows
\begin{align*}
    x &\to (x-x_s)\cdot t_c + (y-y_s)\cdot t_s \\
    y &\to -(x-x_s)\cdot t_s + (y-y_s)\cdot t_c,
\end{align*}
where $x_s$ and $y_s$ are the displacements in the horizontal and vertical
directions, respectively. $t_c$ and $t_s$ are the cosine and sine of the tilt
angle for the element.

\subsection{Electron Lens}
\label{elense}
\subsubsection{Hollow electron lens - uniform annular profile}
For a uniform distribution of the electron beam between $R_1$ and $R_2$, the radial kick can be described by a shape function $f(r)$ and a maximum kick strength $\theta_{\rm max}$:
\begin{equation}
\theta(r)=\frac{f(r)}{(r/R_2)}\cdot \theta_{\rm max}
\end{equation}
with $r=\sqrt{x^2+y^2}$ and $\theta_{\rm max}$ independent of $r$. The shape function $f(r)$ is defined as
\begin{equation}
f(r) = \frac{I(r)}{I_T}=\frac{2\pi}{I_T}\int_{0}^r r\rho(r) dr
\end{equation}
where $I_T$ is the total electron beam current, $I(r)$ is the current enclosed in a radius $r$ and $\rho(r)$ is the electron beam density distribution.

For a uniform profile one then obtains:
\begin{equation}
\begin{cases} 0 &,\quad r< R_1\\
\frac{r^2-R_1^2}{R_2^2-R_1^2} &,\quad R_1 \leq r < R_2\\
1 &,\quad R_2 \leq r
\end{cases}
\end{equation}
and
\begin{equation}
\theta_{\rm max} = \theta(R_2) = \frac{2LI_T(1\pm\beta_e\beta_p)}{4\pi\epsilon_0  \left(B\rho\right)_p\beta_e\beta_p c^2}\cdot\frac{1}{R_2}
\end{equation}
where $L$ is the length of the e-lens, $I_T$ the total electron beam current, $\beta_{e/p}$ the relativistic $\beta$ of electron/proton beam, $B\rho$ the magnetic rigidity, $c$ the speed of light and $\epsilon_0$ the vacuum permittivity. The $\pm$-sign represents the two cases of the electron beam traveling in the direction of the proton beam ($+$) or in the opposite direction ($-$). For hollow electron beam collimation, electron and proton beam travel in the same direction.

The kick in $(x',y')$ can then be expressed as (note $\frac{x}{r}=\cos(\phi),\frac{y}{r}=\sin(\phi)$):
\begin{align}
x'&=x'-\theta_{\rm max}\cdot\frac{r_2}{r^2}\cdot f(r)\cdot x\\
y'&=y'-\theta_{\rm max}\cdot\frac{r_2}{r^2}\cdot f(r)\cdot y
\end{align}
If the electron lens is offset by $(x_{\rm offset},y_{\rm offset})$, the coordinates $(x,y)$ are simply transfered to:
\begin{align}
\tilde x&=x+x_{\rm offset}\\
\tilde y&=y+y_{\rm offset}\\
\tilde r&=\sqrt{\tilde x^2+\tilde y^2}
\end{align}
and the kick is then given by:
\begin{align}
x'&=x'-\theta_{\rm max}\cdot\frac{r_2}{\tilde r^2}\cdot f(\tilde r)\cdot \tilde x\\
y'&=y'-\theta_{\rm max}\cdot\frac{r_2}{\tilde r^2}\cdot f(\tilde r)\cdot \tilde y
\end{align}

\section{Linear optics calculations}
\label{opt}
Optics calculation are needed to study the motion around the closed orbit. By defining $z$ as the vector of $2 k$ coordinates,  
\begin{align}\label{opt:eqn:1}
z&=(z_1,\ldots,z_{2k})^T=(x-x_0,p_x-p_{x0},y-y_0,p_y-p_{y0},\tau-\tau_0,p_t-p_{t0})^T
\end{align}
one can define linear transfer maps (e.g. $M_{1\to 2}$ that propagates coordinates between two points $s_1$, $s_2$) and the one-turn map (e.g. $M_1$ that combines the effects for one turn starting from $s_1$):
\begin{align}\label{opt:eqn:2}
z(s_2)&= M_{1\to 2} z(s_1) & z(C+s_1) &= M_1 z(s_1).
\end{align}
In the following we will describe the optics calculation based on the Ripken formalism described in \cite{willeke88}. A good summary is also given in the MAD8 physics manual \cite{mad8phys}.

\subsection{Diagonalisation of one-turn matrix}
\label{opt:sec:1}
Since the matrices derive from symplectic maps, the eigenvalue spectrum of the one-turn map $M$ consists of 2$k$ distinct eigenvalues and linearly independent eigenvectors. In addition, for the motion to be stable the eigenvalues $\lambda_k^{\pm}$ with eigenvectors $v_k^{\pm}$ have to be complex \cite{willeke88}:
\begin{align}
M v_k^\pm  =  \lambda_k^\pm v_k^\pm, \ k=1,\ldots, 3 \\
v_k^+=(v_k^-)^*, \quad \lambda_k^+=(\lambda_k^-)^*, \quad |\lambda_k^{\pm}|=1
\end{align}
As the eigenvectors are linearly independent $M$ can be diagonalized with
\begin{align}
M &= V \Lambda V^{-1},
\end{align}
where $V$ consists of the eigenvectors and $\Lambda$ of the eigenvalues:
\begin{align}
V=&\left(
\begin{array}{cccc}
v^+_{1,1} & v^-_{1,1} & \cdots & v^-_{3,1}\\
v^+_{1,2} & v^-_{1,2} & \cdots & v^-_{3,2}\\
\vdots    & \vdots    & \vdots & \vdots \\
\end{array}
\right)  &
\Lambda=&\left(
\begin{array}{cccc}
\lambda^+_1 &    & &\\
& \lambda^-_1 & &\\
& & \ddots & \\
& & & \lambda^-_3
\end{array}
\right)
\end{align}
for which $v^{\pm}_{i,j}$ is the component $j$ of eigenvector $v_i^{\pm}$.

The same calculation can be carried out with real numbers by the following definitions:
\begin{align}\
v_k^\pm &= a_k\pm ib_k, & \lambda_k^\pm &= \cos \mu_k \pm i \sin \mu_k, &
\mu_k, a_k, b_k \in \mathbb{R}
\end{align}
such that:
\begin{align}\label{opt:eqn:1:1}
M=&W R W^{-1}
\end{align}
with
\begin{align}
R=R(\mu_k)=&\left(
\begin{array}{ccccc}
\cos \mu_1 & \sin \mu_1  &  & &\\
-\sin \mu_1 & \cos \mu_1 &  & &\\
&    & \ddots & &\\
& & & \cos \mu_3 & \sin \mu_3 \\
& & & -\sin \mu_3 & \cos \mu_3 \\
\end{array}
\right), \\
W=&\left(
\begin{array}{ccccc}
a_{1,1} & b_{1,1} & \cdots & a_{3,1} & b_{3,1} \\
a_{1,2} & b_{1,2} & \cdots & a_{3,2} & b_{3,2} \\
\vdots    & \vdots    & \vdots & \vdots\\
a_{1,6} & b_{1,6} & \cdots & a_{3,6} & b_{3,6} \\
\end{array}
\right)
\end{align}
Usually $\mu_k$ is written as $\mu_k=2\pi Q_k$, where $Q_k$ is then the tune of the mode $k$.
\subsection{Normalisation of eigenvectors}
\label{opt:sec:2}
By convention, the eigenvectors and values are normalized, sorted and rotated so that the following three conditions are fulfilled:
\begin{enumerate}
\item Plane 1 is associated with the horizontal, plane 2 with the vertical and plane 3 with the longitudinal plane. This is achieved by first normalizing the eigenvectors $v_k^{\pm}$ and then sorting them so that:
\begin{align}\label{opt:eqn:2:1}
|v_{j,2j-1}^{+}| =|v_{j,2j-1}^{-}| = \max_{k=1,2,3} v_{k,j}, \quad j=1,\ldots, 3
\end{align}
\item The eigenvectors are then rotated with a phase term $\psi_k$
\begin{align}
v_k& \to v_k \exp(i \psi_k) 
\end{align}
such that
\begin{align}\label{opt:eqn:2:2}
\mathrm{angle}(v_{k,2k-1}^{+})=0 \leftrightarrow \psi_k=-\mathrm{angle}(v_{k,2k-1}^{+})
\end{align}
In real space, Eqn.~\ref{opt:eqn:2:1} and \ref{opt:eqn:2:2} then become equivalent to:
\begin{align}
|a_{j,2j-1}| &=\max_{k=1,2,3} |a_{k,j}|,& b_{j,2 j-1}&=0, & j=1,\ldots, 3
\end{align}
This has the effect that a particle with $x=0$ is transformed to $\tilde x$ in the normalized phase space.
\item The sign of $b_{k,j}$ is fixed by the symplectic condition on $W$
\begin{align}
W^T S W = S
\end{align}
with $S$ defined as
\begin{align}
S&=\left(
\begin{array}{ccc}
0 & 1  &  \\
-1 & 0  &  \\
&    & \ddots \\
\end{array} 
\right)
\end{align}
which is equivalent to:
\begin{align}\label{opt:eqn:2:3}
a_k^T \cdot S \cdot b_k &=1, \quad b_k^T \cdot S \cdot a_k =-1, & \text{ for } k=l\nonumber\\
a_k^T \cdot S \cdot b_l &=0, & \text{ for }  k\not=l\\
a_k^T \cdot S \cdot a_l &=0, \quad b_k^T \cdot S \cdot b_l =0,  & k,l=1,\ldots,3 \nonumber
\end{align}
Eqn.~\ref{opt:eqn:2:3} yields that in phase space $a_k$ is thus obtained by an anticlockwise rotation of $b_k$ by $\pi/2$ and a scaling of its length with $|a_k|=\frac{1}{|b_k|}$.
\end{enumerate}
\subsection{Conversion to normalized coordinates}
\label{opt:sec:3}
We will show in the following that in the normalized phase space the propagation of particle coordinates $z(s)$ from $s_1$ to $s_2$ is just a rotation by an angle $\phi_k$ in the $k=1,\ldots,3$ planes, while the amplitude $I_k$ and initial phase $\phi_{k,0}$ stay constant, explicitly $z(s)$ is then given by:
\begin{align}\label{opt:eqn:3:3}
z(s)=\sum_{k=1}^3 \sqrt{2I_k}\left(
a_k(s) \cos \left(\phi_{k,0} + \phi_k(s)\right) -
b_k(s) \sin \left(\phi_{k,0} + \phi_k(s)\right)
\right) 
\end{align}
and
\begin{align}
z(s_2)&=W(s_2)R(\phi_k)W(s_1)^{-1}z(s_1), \\
&\hspace{30pt} \text{ with } \phi_k=\phi_k(s_2)-\phi_k(s_1)\nonumber
\end{align}
This implies that one turn is simply a rotation by $\phi_k=2\pi Q_k$ where $Q_k$ is the tune of the mode $k$. In the transverse plane the tune ($Q_{I,II}$) is usually positive and the particles rotate clockwise, while in the longitudinal plane the tune ($Q_{III}$) is negative above $\gamma_T$ leading to an anticlockwise rotation.

For the derivation the following steps are needed:
\begin{enumerate}
\item The effect of one turn on the normalized variable $\tilde z(s)=W^{-1}(s) z(s)$ is a rotation:
\begin{align}\label{opt:eqn:3:1}
\tilde z(C+s) = W^{-1}z(s+C)\overset{(\rm Eqn.\ref{opt:eqn:1:1})}{=}W^{-1}WRW^{-1}z(s)= R\tilde z(s),
\end{align}
As $M$ and $R$ are symplectic also $W$ is symplectic, and its inverse is thus given by $S^{-1}W^{T}S$, explicitly:
\begin{align}
W^{-1}&=
\left(
\begin{array}{cccccc}
b_{12} & - b_{11} &   b_{14} & - b_{13} &   b_{16} & - b_{15}\\
- a_{12} &   a_{11} & - a_{14} &   a_{13} & - a_{16} &   a_{15}\\
b_{22} & - b_{21} &   b_{24} & - b_{23} &   b_{26} & - b_{25}\\
- a_{22} &   a_{21} & - a_{24} &   a_{23} & - a_{26} &   a_{25}\\
b_{32} & - b_{31} &   b_{34} & - b_{33} &   b_{36} & - b_{35}\\
- a_{32} &   a_{31} & - a_{34} &   a_{33} & - a_{36} &   a_{35}\\
\end{array}
\right)
\end{align}
\item The one-turn map and $W$-matrix can be propagated from $s_1$ to $s_2$ by
\begin{align}
M_2&=M_{1 \to 2} M_1 M^{-1}_{1 \to 2}  &
W_2&=M_{1 \to 2} W_1
\end{align}
As Eqn.~\ref{opt:eqn:3:1} represents a similarity transformation, the eigenvalues are thus independent of the position $s$ and as the rotation matrix $R$ consists of the eigenvalues of $M$, the angle of the rotation $\mu_k=2\pi Q_k$ is thus also independent of $s$.

\item As Eqn.~\ref{opt:eqn:1:1} represents a basis transformation from the standard $\mathbb{R}^2$ basis to the eigenvector basis, the vectors $a_k$ and $b_k$ are projected onto (Eqn.~\ref{opt:eqn:2:3}):
\begin{align}\label{opt:eqn:3:2}
\tilde a_1=W^{-1}a_1&=-SW^TSa_1\nonumber\\
&=-S(a_1Sa_1,b_1Sa_1,\ldots,b_3Sa_1)^T=(1,0,\ldots,0)\nonumber\\
\tilde b_1=W^{-1}b_1&=-SW^TSb_1\nonumber\\
&=-S(a_1Sb_1,b_1Sb_1,\ldots,b_3Sb_1)^T=(0,1,\ldots,0)\\
\cdots & \nonumber\\
\tilde b_3=W^{-1}b_3&=-SW^TSb_3\nonumber\\
&=-S(a_1Sb_3,b_1Sb_3,\ldots,b_3Sb_3)^T=(0,0,\ldots,1)\nonumber
\end{align}
in the normalized phase space.
\item From Eqn.~\ref{opt:eqn:3:1} it follows that the amplitude $I_k$ and initial phase $\phi_{k0}$ of $\tilde z=W^{-1}z=(\tilde z_{a_1},\tilde z_{b_1},\ldots,\tilde z_{b_3})$ 
\begin{align}
I_k&=\frac{(\tilde z_{a_k})^2 +(\tilde z_{b_k})^2}{2}, \quad k=1,\ldots,3\label{opt:eqn:20a}\\
\tan\phi_{k0}&=-\frac{\tilde z_{b_k}}{\tilde z_{a_k}} \label{opt:eqn:20b}
\end{align}
are constants of the motion.
%, which is illustrated in Fig.~\ref{opt:fig:1}.
%\begin{figure}[h]
%	\centering
%	\includegraphics{normalized_phase_space_cropped.pdf}
%	\caption{Normalized phase space.\label{opt:fig:1}}
%\end{figure}
The initial phase is defined with a minus sign in view of the definition of the Twiss parameters, where the initial phase is then added (and not subtracted) to the phase advance. The components of $\tilde z$ are then explicitly given by:
\begin{align}
\tilde z_{a_k}&= \sum_{j=1}^3  b_{k,2j} z_{2j-1}- b_{k,2j-1} z_{2j}, \quad k=1,\ldots,3\\
\tilde z_{b_k}&= \sum_{j=1}^3 a_{k,2j-1} z_{2j}- a_{k,2j} z_{2j-1}, \quad k=1,\ldots,3.
\end{align}
An arbitrary vector $z(s)$ can thus be written in the following form:
\begin{align}
 z(s)&=W(s)\tilde z(s)\nonumber\\
 &=W(s)\left(\sum_{k=1}^{3}\tilde z_{a_k}\tilde a_k + \tilde z_{b_k}\tilde b_k\right)\nonumber\\
 &=\sum_{k=1}^{3}\tilde z_{a_k}W(s)\tilde a_k + \tilde z_{b_k}W(s)\tilde b_k\overset{\rm Eqn.~\ref{opt:eqn:3:2}}{=}\sum_{k=1}^{3}\tilde z_{a_k}a_k + \tilde z_{b_k}b_k\nonumber\\
 &\overset{\rm Eqns.~\ref
 	{opt:eqn:20a},\ref{opt:eqn:20b}}=\sum_{k=1}^{3}\sqrt{2I_k}\left(a_k\cos{\phi_{k0}}-b_k\sin{\phi_{k0}}\right)
\end{align}
\end{enumerate}

\subsection{Twiss parameters}
\label{opt:sec:4}
In the following the parameter $k$ will always be used for the mode $k$ and the parameter $j=1,2,3$ for the horizontal ($x,p_x$), vertical ($y,p_y$) and longitudinal plane $(\sigma,\delta)$ in the phase space. $z_{2j-1}$ then stands for the coordinates $(x,y,\sigma$) and $z_{2j}$ for $(p_x,_y,\delta$).

The Twiss parameters can be introduced by writing the components of the eigenvector basis $(a_k(s),b_k(s))$ as the product of two envelope functions $\sqrt{\beta_{k,j}(s)}$, $\sqrt{\gamma_{k,j}(s)}$ and phase functions $\phi_{k,j}(s)$, $\bar\phi_{k,j}(s)$, also called Twiss parameters or lattice functions, with
\begin{align}
a_{k,2j-1}(s)&=\sqrt{\beta_{k,j}(s)}\cos{\phi_{k,j}(s)},\nonumber\\ b_{k,2j-1}(s)&=\sqrt{\beta_{k,j}(s)}\sin{\phi_{k,j}(s)}, \ k,j=1,\ldots,3, \label{opt:eqn:4:1}\\
a_{k,2j}(s)&=\sqrt{\gamma_{k,j}(s)}\cos{\bar\phi_{k,j}(s)}, \nonumber\\
b_{k,2j}(s)&=\sqrt{\gamma_{k,j}(s)}\sin{\bar\phi_{k,j}(s)}, \ k,j=1,\ldots,3 \label{opt:eqn:4:2}
\end{align}
where $\beta_{k,j}(s), \alpha_{k,j}(s), \gamma_{k,j}(s)$ represent the projection of the ellipse of mode $k$ on the plane of coordinates $z_{2k-1}-z_{2k}$. 
%(see Fig.~\ref{opt:fig:2})
%\begin{figure}[!ht]
%	\centering
%	\includegraphics[width=1.0\linewidth]{ripken_phase_space_ellipse.png}
%	\caption{Projection of lattice function in the $z-z'$ plane.\label{opt:fig:2}}
%\end{figure}

Using Eqns.~\ref{opt:eqn:3:3}, \ref{opt:eqn:4:1}, \ref{opt:eqn:4:2} and $\cos(x+y)=\cos x\cos y-\sin x\sin y$, the coordinates $z(s)$ can be expressed by:
\begin{align}
z_{2j-1}(s)&=\sum_{k=1}^3 \sqrt{2I_k
\beta_{k,j}(s)}\cos{(\phi_{k,j}(s)+\phi_{k,0})}\\
z_{2j}(s)&=\sum_{k=1}^3 \sqrt{2I_k
\gamma_{k,j}(s)}\cos{(\bar\phi_{k,j}(s)+\phi_{k,0})}, \ j=1,\ldots,3
\end{align}
Conversely the lattice functions can also be expressed by $a_k$ and $b_k$ with
\begin{align}
\beta_{k,j}(s)&=a_{k,2j-1}(s)^2 +b_{k,2j-1}(s)^2 \\
\alpha_{k,j}(s)&=- a_{k,2j-1}(s)a_{k,2j}(s) -b_{k,2j-1}(s)b_{k,2j}(s) \\
\gamma_{k,j}(s)&=a_{k,2j}(s)^2 +b_{k,2j}(s)^2,
\end{align}
The well known relations between the lattice functions
\begin{align}
\sum_{j=1}^3\beta_{k,j}\phi_{k,j}'&=1 \\
\gamma_{k,j}&=\frac{\beta_{k,j}^2\phi_{k,j}'^2+\alpha_{k,j}^2}{\beta_{k,j}}, \text{ with  }\\
\alpha_{k,j}&:=-\frac{1}{2}\beta_{k,j}'
\end{align}
can then be derived with the help of the normalization condition (Eqn.~\ref{opt:eqn:2:3})
\begin{align}
a_k^TSb_k=1
\end{align}
by the following steps:
\begin{enumerate}
\item As $x'=\frac{dx}{ds},\ y'=\frac{dy}{ds}$ and $\delta=\frac{d\sigma}{ds}$ the following relations hold also for $a_k$ and $b_k$:
\begin{align}
a_{k,2j}=a_{k,2j-1}'&=\frac{d}{ds}(a_{k,2j-1}), \\
b_{k,2j}=b_{k,2j-1}'&=\frac{d}{ds}(b_{k,2j-1}),\ k,j=1,\ldots,3 
\end{align}
\item The normalization condition Eqn.~\ref{opt:eqn:2:3} can then be written as
\begin{align}
a_k^TSb_k&=\sum_{j=1}^3\sqrt{\beta_{k,j}}\cos{\phi_{k,j}}\left(\sqrt{\beta_{k,j}}\sin{\phi_{k,j}}\right)'\nonumber\\
& \qquad -\left(\sqrt{\beta_{k,j}}\cos{\phi_{k,j}}\right)'\sqrt{\beta_{k,j}}\sin{\phi_{k,j}}\nonumber\\
&=\sum_{j=1}^3\beta_{k,j}\phi_{k,j}'\nonumber\\
&=1 \label{opt:eqn:4:3}
\end{align}
Note that Eqn.~\ref{opt:eqn:4:3} yields the the following relation between the phase advance $\phi$ and $\beta$ in 2D:
\begin{align}
\phi(s)=\phi(0)+\int_{s_0}^s\frac{1}{\beta(\bar s)}d\bar s
\end{align}
\item Using the abbreviation $\alpha_{k,j}:=-\frac{1}{2}\beta_{k,j}$, one finds for each mode $k$ and plane $j$
\begin{align}
\sqrt{\gamma_{k,j}}\cos{\phi_{k,j}}&=a_{k,2j}=a_{k,2j-1}'=(\sqrt{\beta_{k,j}}\cos{\phi_{k,j}})' &\quad (1)\nonumber\\
\sqrt{\gamma_{k,j}}\sin{\phi_{k,j}}&=b_{k,2j}=b_{k,2j-1}'=(\sqrt{\beta_{k,j}}\sin{\phi_{k,j}})' &\quad (2)\nonumber\\
\overset{(1)^2+(2)^2}{\Rightarrow} \gamma_{k,j}&=\frac{\beta_{k,j}^2\phi_{k,j}'^2+\alpha_{k,j}^2}{\beta_{k,j}}, \quad k,j=1,\ldots,3 &
\end{align}
which simplifies in the 2D case to:
\begin{align}
\gamma\overset{\rm Eqn.~\ref{opt:eqn:4:3}}{=}\frac{1+\alpha^2}{\beta}
\end{align}
\end{enumerate}




\chapter{Xfields}


\section{Fields generated by a bunch of particles}



We assume that the bunch travels rigidly along $s$ with velocity $\beta_0 c$:
\begin{align}
&\rho\left(x, y, s, t\right) = \rho_0\left(x, y, s - \beta_0 ct\right) \label{rhorho0}\\
&\textbf{J}\left(x, y, s, t\right) = \beta_0c\, \rho_0\left(x, y, s - \beta_0 ct\right)  \hat{\textbf{i}}_s \label{JJ0}
\end{align}

We define an auxiliary variable $\zeta$ as the position along the bunch:
\begin{equation}
\zeta = s -\beta_0 c t \, .\label{zetadef}
\end{equation}
We call $K$ the lab reference frame in which we have defined all equations above, and we introduce a boosted frame $K'$ moving rigidly with the reference particle.
The coordinates in the two systems are related by a Lorentz transformation~\cite{jackson}:
\begin{align}
ct' &= \gamma_0 \left(ct -\beta_0 s \right)\label{lorA}\\
x' &= x\label{lorX}\\
y' &= y\label{lorY}\\
s' &= \gamma_0 \left(s -\beta_0 ct \right) = \gamma_0 \zeta\label{lorB}
\end{align}
The  corresponding inverse transformation is:
\begin{align}
ct &= \gamma_0 \left(ct' +\beta_0 s' \right)\label{lorC}\\
x &= x'\label{lorXinv}\\
y &= y'\label{lorYinv}\\
s &= \gamma_0 \left(s' +\beta_0 ct' \right)\label{lorD}
\end{align}



The quantities $\left(c \rho, J_x, J_y, J_s\right)$ form a Lorentz 4-vector and therefore they are transformed between $K$ and $K'$ by relationships similar to the Eqs.~\ref{lorA}-\ref{lorY}~\cite{jackson}:
\begin{align}
c\rho' \left(\textbf{r'}, t'\right)\ &= \gamma_0 \left[c \rho  \left(\textbf{r}\left(\textbf{r'}, t'\right), t\left(\textbf{r'}, t'\right)\right) -\beta_0 J_s \left(\textbf{r}\left(\textbf{r'}, t'\right), t\left(\textbf{r'}, t'\right)\right) \right]\label{lorrho}\\
J_s' \left(\textbf{r'}, t'\right)\ &= \gamma_0 \left[J_s  \left(\textbf{r}\left(\textbf{r'}, t'\right), t\left(\textbf{r'}, t'\right)\right) -\beta_0 c \rho \left(\textbf{r}\left(\textbf{r'}, t'\right), t\left(\textbf{r'}, t'\right)\right) \right]\label{lorjs}
\end{align}
where the transformations $\textbf{r}\left(\textbf{r'}, t'\right)$ and $t\left(\textbf{r'}, t'\right)$ are defined by Eqs.~\ref{lorC} and~\ref{lorD} respectively. The transverse components $J_x$ and $J_y$ of the current vector are invariant for our transformation, and are anyhow zero in our case.

Using Eq.\,\ref{JJ0} these become:
\begin{align}
\rho' \left(\textbf{r'}, t'\right)\ &= \frac{1}{\gamma_0}\rho\left(\textbf{r}\left(\textbf{r'}, t'\right), t\left(\textbf{r'}, t'\right)\right)
\\
J_s' \left(\textbf{r'}, t'\right)\ & = 0
\end{align}

Using Eqs.~\ref{rhorho0} and~\ref{lorC}-\ref{lorYinv}, we obtain:
\begin{equation}
\rho  \left(x', y', s(s', t'), t(s', t')\right) = \rho_0  \left(x', y', s(s', t') - \beta_0 c\,t(s', t')\right)
\end{equation}

From Eq.~\ref{lorB} we get:
\begin{equation}
s(s', t')- \beta_0 c\,t(s', t') = \frac{s'}{\gamma_0} 
\end{equation}
where the coordinate $t' $ has disappeared.

We can therefore write:
\begin{equation}
\rho' \left(x', y', s', t'\right) =   \frac{1}{\gamma_0} \rho_0  \left(x', y',  \frac{s'}{\gamma_0}\right)\label{rhoprimerho0}
\end{equation}

The electric potential in the bunch frame is solution of Poisson's equation:

\begin{equation}
\frac{\partial^2 \phi'}{\partial x'^2} +  \frac{\partial^2 \phi'}{\partial y'^2}+  \frac{\partial^2 \phi'}{\partial s'^2}= -\frac{\rho' (x', y', s')}{\varepsilon_0}
\end{equation}

From Eq.~\ref{rhoprimerho0} we can write:
\begin{equation}
\frac{\partial^2 \phi'}{\partial x'^2} +  \frac{\partial^2 \phi'}{\partial y'^2}+  \frac{\partial^2 \phi'}{\partial s'^2}= -\frac{1}{\gamma_0\varepsilon_0}  \rho_0 \left(x', y', \frac{s'}{\gamma_0}\right)\label{poissrho0}
\end{equation}

We now make the substitution:
\begin{equation}
\zeta = \frac{s'}{\gamma_0} \label{subst}
\end{equation}
obtained from Eq.~\ref{lorB}, which allows to rewrite Eq.~\ref{poissrho0} as:
\begin{equation}
\frac{\partial^2 \phi'}{\partial x^2} +  \frac{\partial^2 \phi'}{\partial y^2}+  \frac{1}{\gamma_0^2}\frac{\partial^2 \phi'}{\partial \zeta^2}=  -\frac{1}{\gamma_0\varepsilon_0}{\rho}_0 \left(x, y,\zeta\right) \label{modifpoiss}
\end{equation}
Here we have dropped the ``$'$'' sign from $x$ and $y$ as these coordinates are unaffected by the Lorentz boost.




The quantities $\left( \frac{\phi}{c}, A_x, A_y, A_s\right)$ form a Lorentz 4-vector, so we can write:
\begin{align}
\phi &= \gamma_0 \left( {\phi'} +  \beta_0 c A'_s\right)\\
A_s &= A'_s +\beta_0 \frac{\phi'}{c}
\end{align}
In the bunch frame the charges are at rest therefore $A'_x=A'_y=A'_s=0$ therefore:
\begin{align}
\phi &= \gamma_0 \phi'\label{phiphip}\\
A_s &= \beta_0 \frac{\phi'}{c} =  \frac{\beta_0}{\gamma_0c}\phi
\end{align}

Combining Eq.\,\ref{phiphip} with Eq.\,\ref{modifpoiss} we obtain the equation in $\phi$:
\begin{equation}
\boxed{
\frac{\partial^2 \phi}{\partial x^2} +  \frac{\partial^2 \phi}{\partial y^2}+  \frac{1}{\gamma_0^2}\frac{\partial^2 \phi}{\partial \zeta^2}=  -\frac{1}{\varepsilon_0}{\rho}_0 \left(x, y,\zeta\right)} \label{modifpoiss_zeta}
\end{equation}

\subsection{2.5D approximation}
For large enough values of $\gamma_0$, Eq.~\ref{modifpoiss} can be approximated by:
\begin{equation}
\boxed{
\frac{\partial^2 \phi}{\partial x^2} +  \frac{\partial^2 \phi}{\partial y^2} = -\frac{1}{\varepsilon_0}{\rho}_0 \left(x, y,\zeta\right) }\label{2dpoiss}
\end{equation}
which means that we can solve a simple 2D problem for each beam slice (identified by its coordinate $\zeta$).


\subsection{Modulated 2D}
\label{sec:modulated2d}

Often the beam distribution can be factorized as:
\begin{equation}
\rho_0(x,y,\zeta) = q_0\lambda_0(\zeta)\rho_\perp(x,y) 
\end{equation}
where:
\begin{equation}
\int \rho_\perp(x,y) \,dx\,dy = 1
\end{equation}
and $\lambda_0(z)$ is therefore the bunch line density.

For a bunched beam:
\begin{equation}
\int \lambda_0(z) \,dz = N \label{eq:lamnorm}\\
\end{equation}
where $N$ is the bunch population.

In this case the potential can be factorized as:
\begin{equation}
\phi(x,y,\zeta) = q_0\lambda(\zeta)\phi_\perp(x,y) 
\label{eq:factorized2d}
\end{equation}

where $\phi_\perp(x,y)$ is the solution of the following 2D Poisson equation:
\begin{equation}
\frac{\partial^2 \phi_\perp}{\partial x^2} +  \frac{\partial^2 \phi_\perp}{\partial y^2} = -\frac{1}{\varepsilon_0}{\rho}_\perp \left(x, y\right) \label{2dpoisspeerp}
\end{equation}


%\section{Interaction time}
%In the lab frame the particle moves with speed $\beta$:
%\begin{equation}
%s(t) = \zeta_p +\beta c t
%\end{equation}
%
%In the frame $K'$, the kinematic equation of the particle can be obtained by replacing Eqs.~\ref{lorC} and~\ref{lorD} into Eq.~\ref{st_tau}:
%\begin{equation}
%\gamma_0 \left(s' +\beta_0 ct' \right) = \zeta_p +\beta \gamma_0 \left(ct' +\beta_0 s' \right)
%\end{equation}
%
%Solving for $s'$ we obtain:
%\begin{equation}
%s' = -\beta \gamma c \tau = \gamma \zeta\label{sprimezeta}
%\end{equation}
%Of course for the reference particle we obtain $s' = 0$.
%We observe that \textbf{beam particles are at rest in the reference frame $K'$ and that the distance between them is increased by a factor $\gamma$ with respect to the lab frame $K$}.

%\section{Transverse kick on the beam particle}
%
%We now evaluate the change on the transverse momentum for a beam particle defined in the lab frame by its transverse coordinates $x$ and $y$ and by its delay $\tau$ with respect to the reference particle (or equivalently by its $\zeta$ coordinate, defined by Eq.~\ref{zetadef}).
%
%We have seen that in the frame $K'$ the particle is at rest and has longitudinal coordinate $s' = \gamma \zeta$ (see Eq.~\ref{sprimezeta}). 
%The x' component of the electric field $\textbf{E}'$ acting on P is given by (see Eqs.~\ref{potential} and~\ref{phiphiprime}):
%\begin{equation}
%E'_x = -\frac{\partial \phi'}{\partial x} = -\frac{1}{\gamma_0}\frac{\partial \phi}{\partial x} \label{Exprime}
%\end{equation}
%Again, we have dropped the ``$'$'' sign from $x$ and $y$ as these coordinates are unaffected by the Lorentz boost.
%
%
%The change in the x component of the momentum, which is an invariant for our Lorentz transformation, is given by :
%\begin{equation}
%\Delta P_x = \Delta P'_x = qE'_x T'
%\end{equation}
%
%Using Eqs.~\ref{Exprime} and~\ref{Tprime} we can write:
%\begin{equation}
%\Delta P_x = -\frac{qL}{\beta c} \frac{\partial \phi}{\partial x}\left(x, y,\zeta\right)
%\end{equation}
%
%Normalizing to the momentum of the reference particle:
%
%\begin{equation}
%\Delta p_x = \frac{\Delta P_x} {P}= -\frac{qL}{ m\gamma\beta^2 c^2} \frac{\partial \phi}{\partial x}\left(x, y,\zeta\right)\label{dpx}
%\end{equation}
%
%Similarly, for the $y$-direction we can write: 
%\begin{equation}
%\Delta p_y = \frac{\Delta P_y} {P}= -\frac{qL}{ m\gamma\beta^2 c^2} \frac{\partial \phi}{\partial y}\left(x, y,\zeta\right)\label{dpy}
%\end{equation}

\section{Lorentz force}
We now compute the Lorentz force on the particles moving in the longitudinal directions, including particles of the bunch itself (space charge forces) and particles of a colliding bunch moving in the opposite directions (beam-beam forces).
The angles of such test particles are neglected as done in the usual thin-lens approximation. Therefore the velocity of a test particle can be written as:
\begin{equation}
\textbf{v} = \beta c\, \hat{\textbf{i}}_s
\end{equation}

The Lorenz force can be written as:
\begin{equation}
\begin{split}
\textbf{F} &=q \left( -\nabla \phi -\frac{\partial \textbf{A}}{\partial t}
 + \beta c \ \hat{\textbf{i}}_s \times {\left(\nabla \times \textbf{A} \right)} \right)\\
 &=q \left( -\nabla \phi -\frac{\beta_0}{\gamma_0 c}\frac{\partial \phi}{\partial t}\hat{\textbf{i}}_s
 + \beta c \ \hat{\textbf{i}}_s \times {\left(\nabla \times \textbf{A} \right)} \right)
 \end{split}
\end{equation}

We compute the vector product:
\begin{align}
\begin{split}
\hat{\textbf{i}}_s \times \left(\nabla \times \textbf{A}\right) &= \left(\frac{\partial A_s}{\partial x} - \frac{\partial A_x}{\partial s} \right) \hat{\textbf{i}}_x + \left(\frac{\partial A_s}{\partial y} - \frac{\partial A_y}{\partial s} \right) \hat{\textbf{i}}_y\\
 &= \left(\frac{\partial A_s}{\partial x} - \frac{\partial A_x}{\partial s} \right) \hat{\textbf{i}}_x + \left(\frac{\partial A_s}{\partial y} - \frac{\partial A_y}{\partial s} \right) \hat{\textbf{i}}_y + \underbrace{\left(\frac{\partial A_s}{\partial s} - \frac{\partial A_s}{\partial s} \right)}_{=0} \hat{\textbf{i}}_s\\
 &= \nabla A_s - \frac{\partial \textbf{A}}{\partial s} 
\end{split} 
\end{align}

We replace:
\begin{equation}
\textbf{F} 
=q \left( -\nabla \phi -\frac{\beta_0}{\gamma_0 c}\frac{\partial \phi}{\partial t}\hat{\textbf{i}}_s
 + \beta  \beta_0\nabla \phi - \frac{\beta \beta_0}{\gamma_0} \frac{\partial \phi}{\partial s} \hat{\textbf{i}}_s
  \right)
\end{equation}

The potentials will have the same form as the sources (this can be shown explicitly using the Lorentz transformations):
\begin{equation}
\phi(x, y, s, t) = \phi\left(x, y, t - \frac{s}{\beta_0 c}\right)
\end{equation}
For a function in this form we can write:
\begin{equation}
 \frac{\partial \phi}{\partial s} = 
\frac{\partial}{\partial\zeta} 
 = -\frac{1}{\beta_0 c}\frac{\partial \phi}{\partial t} \label{derder}
\end{equation}


obtaining:
\begin{equation}
\textbf{F} 
=q \left( -\nabla \phi +\frac{\beta_0^2}{\gamma_0}\frac{\partial \phi}{\partial \zeta}\hat{\textbf{i}}_s
 + \beta  \beta_0\nabla \phi - \frac{\beta \beta_0}{\gamma_0} \frac{\partial \phi}{\partial \zeta} \hat{\textbf{i}}_s
  \right)
\end{equation}


Reorganizing:
\begin{equation}
\textbf{F} 
=  -q(1-\beta  \beta_0)\nabla \phi -\frac{\beta_0(\beta-\beta_0)}{\gamma_0}\frac{\partial \phi}{\partial \zeta}\hat{\textbf{i}}_s
\end{equation}

Writing the dependencies explicitly:
\begin{align}
F_x(x, y, \zeta(t)) &=  -q(1-\beta  \beta_0) \frac{\partial \phi}{\partial x}(x, y, \zeta(t))\label{eq:forcex}\\
F_y(x, y, \zeta(t)) &=  -q(1-\beta  \beta_0) \frac{\partial \phi}{\partial y}(x, y, \zeta(t))\label{eq:forcey}\\
F_z(x, y, \zeta(t)) &=  -q\left(1-\beta  \beta_0 -\frac{\beta_0(\beta-\beta_0)}{\gamma_0}\right) \frac{\partial \phi}{\partial \zeta}(x, y, \zeta(t))\label{eq:forcez}
\end{align}
where $\zeta(t)$ is the position of the particle within the bunch.


\section{Space charge}

Over the single interaction we neglect the particle slippage\footnote{In any case one would need to take into account also the dispersion in order to have the right slippage.}:
\begin{align}
&\beta = \beta_0\\
&\zeta(t) = \zeta
\end{align}

This gives the following simplification of Eqs.\,\eqref{eq:forcex}\,-\,\eqref{eq:forcez}:
\begin{align}
F_x(x, y, \zeta) &=  -q(1-\beta_0^2) \frac{\partial \phi}{\partial x}(x, y, \zeta)\\
F_y(x, y, \zeta) &=  -q(1-\beta_0^2) \frac{\partial \phi}{\partial y}(x, y, \zeta)\\
F_z(x, y, \zeta) &=  -q (1-\beta_0^2) \frac{\partial \phi}{\partial \zeta}(x, y, \zeta)
\end{align}

In this way the force over the single interaction becomes independent on time and therefore we can compute the kicks simply as:
\begin{equation}
\Delta \textbf{P} = \frac{L}{\beta_0 c}\textbf{F} 
\end{equation}
where $L$ is the portion of the machine on which we want to compute the e-cloud interaction.

The kicks on the normalized momenta can be expressed as (recalling that $P_0=m_0\beta_0\gamma_0c$):

\begin{align}
&\boxed{
\Delta p_x = \frac{m_0}{m}\frac{\Delta P_x} {P_0}= -\frac{qL (1-\beta_0^2)}{ m\gamma_0\beta_0^2 c^2} \frac{\partial \phi}{\partial x}\left(x, y,\zeta\right)}\label{dpx}\\
&\boxed{
\Delta p_y = \frac{m_0}{m}\frac{\Delta P_y} {P_0}= -\frac{qL (1-\beta_0^2)}{ m\gamma_0\beta_0^2 c^2} \frac{\partial \phi}{\partial y}\left(x, y,\zeta\right)}\label{dpy}\\
&\boxed{
\Delta \delta \simeq \Delta p_z = \frac{m_0}{m}\frac{\Delta P_z} {P_0}= -\frac{qL (1-\beta_0^2)}{ m\gamma_0\beta_0^2 c^2} \frac{\partial \phi}{\partial \zeta}\left(x, y,\zeta\right)}
\label{dpz}
\end{align}

If the beam includes particles of different species (tracking of fragments), note that here $q$ and $m$ refer to the individual particle while $m_0$ is the mass of the reference particle.



In the modulated 2D case (see Sec.\,\ref{sec:modulated2d} and in particular Eq.\,\ref{eq:factorized2d}), the kick can be expressed as:
\begin{align}
&\boxed{
\Delta p_x = \frac{m_0}{m}\frac{\Delta P_x} {P_0}= -\frac{qq_0L (1-\beta_0^2)}{ m\gamma_0\beta_0^2 c^2} \,\lambda_0(\zeta)\,\frac{\partial {\phi_\perp}}{\partial x}\left(x, y\right)}\label{dpx_mod}\\
&\boxed{
\Delta p_y = \frac{m_0}{m}\frac{\Delta P_y} {P_0}= -\frac{qq_0L (1-\beta_0^2)}{ m\gamma_0\beta_0^2 c^2} \,\lambda_0(\zeta)\, \frac{\partial{\phi_\perp}}{\partial y}\left(x, y\right)}\label{dpy_mod}\\
&\boxed{
\Delta \delta \simeq \Delta p_z = \frac{ m_0} {m}\frac{\Delta P_z} {P_0}= -\frac{qq_0L (1-\beta_0^2)}{ m\gamma_0\beta_0^2 c^2} \,\frac{d\lambda_0}{d\zeta}(\zeta)\,{\phi_\perp}\left(x, y\right)}\label{dpz_mod}
\end{align}

\section{Beam-beam interaction}

We consider a test particle moving in the opposite direction with velocity:
\begin{align}
\textbf{v}_W = -\beta_{0W} c\, \hat{\textbf{i}}_s\\
s_W(t) = -\beta_{0W} ct
\end{align}
Equations\,\eqref{eq:forcex}\,-\,\eqref{eq:forcez} become:
\begin{align}
F_x(x, y, \zeta_W(t)) &=  -q(1+\beta_{0W}  \beta_{0s}) \frac{\partial \phi}{\partial x}(x, y, \zeta_W(t)) \label{eq:bbfgenx}\\
F_y(x, y, \zeta_W(t)) &=  -q(1+\beta_{0W}  \beta_{0S}) \frac{\partial \phi}{\partial y}(x, y, \zeta_W(t))\label{eq:bbfgeny}\\
F_z(x, y, \zeta_W(t)) &=  -q\left(1+\beta_{0W}  \beta_{0S} -\frac{\beta_{0S}(\beta_{0W}+\beta_{0S})}{\gamma_0}\right) \frac{\partial \phi}{\partial \zeta}(x, y, \zeta_W(t))\label{eq:bbfgenz}
\end{align}
where we have used the the subscript $S$ (strong) for the bunch generating the fields, and the subscript $W$ (weak) for the test particle. 

$\zeta_W(t)$ is the position of the test particle within the bunch generating the fields: 
\begin{equation}
\zeta_W(t)= s_W(t) -\beta_{0S} c t  = -(\beta_{0W}+\beta_{0S})ct
\label{eq:zetaw}
\end{equation}

In modulated-2D case (Eq.\,\ref{eq:factorized2d}), Eqs.\,\eqref{eq:bbfgenx}\,-\,\eqref{eq:bbfgeny} become:
\begin{align}
F_x(x, y, \zeta_W(t)) &=  -q q_{0S} (1+\beta_{0W}  \beta_{0s})
\lambda_{0S}(\zeta_W(t))
 \frac{\partial \phi_\perp}{\partial x}(x, y ) \\
F_y(x, y, \zeta_W(t)) &=  -qq_{0S}  (1+\beta_{0W}  \beta_{0s})
\lambda_{0S}(\zeta_W(t))
 \frac{\partial \phi_\perp}{\partial y}(x, y ) \\
F_z(x, y, \zeta_W(t)) &=  -qq_{0S}\left(1+\beta_{0W}  \beta_{0S} -\frac{\beta_{0S}(\beta_{0W}+\beta_{0S})}{\gamma_0}\right) \frac{d \lambda_{0S}}{d \zeta}(\zeta_W(t)) \phi_\perp(x, y)
\end{align}

The change in momentum for the test particle is given by:
\begin{equation}
\Delta \textbf{P} = \int_{-\infty}^{+\infty} \textbf{F}(t) \,dt
\end{equation}
Therefore:
\begin{align}
\Delta P_x(x, y, \zeta_W(t)) &=  -qq_{0S} N_S (1+\beta_{0W}  \beta_{0s})
\frac{\partial \phi_\perp}{\partial x}(x, y ) \int_{-\infty}^{+\infty}\lambda_{0S}(\zeta_W(t)) \,dt\\
\Delta P_y(x, y, \zeta_W(t)) &=  -qq_{0S} N_S (1+\beta_{0W}  \beta_{0s})
 \frac{\partial \phi_\perp}{\partial y}(x, y ) \int_{-\infty}^{+\infty}\lambda_{0S}(\zeta_W(t)) \,dt\\
\Delta P_z(x, y, \zeta_W(t)) &=  -qq_{0S}\left(1+\beta_{0W}  \beta_{0S} -\frac{\beta_{0S}(\beta_{0W}+\beta_{0S})}{\gamma_0}\right) \phi_\perp(x, y) \int_{-\infty}^{+\infty}\frac{d \lambda_{0S}}{d \zeta}(\zeta_W(t)) \, dt
\end{align}

Using Eq.\,\eqref{eq:zetaw} and Eq.\,\eqref{eq:lamnorm} we can write:
\begin{equation}
\int_{-\infty}^{+\infty}\lambda_{0S}(\zeta_W(t)) \,dt 
=\frac{1}{(\beta_{0W}+\beta_{0S})c}\int_{-\infty}^{+\infty}\lambda_{0S}(\zeta) \,d\zeta = \frac{N_S}{(\beta_{0W}+\beta_{0S})c}
\end{equation}

Similarly, for a bunched beam:
\begin{equation}
\int_{-\infty}^{+\infty}
\frac{d \lambda_{0S}}{d \zeta}(\zeta_W(t)) \,dt 
=\frac{1}{(\beta_{0W}+\beta_{0S})c}\int_{-\infty}^{+\infty}\frac{d \lambda_{0S}}{d \zeta} \,d\zeta = \frac{ \lambda_{0S}(+\infty)-\lambda_{0S}(-\infty)}{(\beta_{0W}+\beta_{0S})c} = 0
\end{equation}

From which we can write:
\begin{align}
&\boxed{
\Delta p_x = \frac{m_0}{m}\frac{\Delta P_x} {P_0}= -\frac{qq_{0S} N_S 
}{m\beta_{0W}\gamma_{0W}c^2}
\frac{(1+\beta_{0W}  \beta_{0s})}{(\beta_{0W}+\beta_{0S})}
\frac{\partial \phi_\perp}{\partial x}(x, y )}\\
&\boxed{
\Delta p_y = \frac{m_0}{m}\frac{\Delta P_y} {P_0}= -\frac{qq_{0S} N_S 
}{m\beta_{0W}\gamma_{0W}c^2}
\frac{(1+\beta_{0W}  \beta_{0s})}{(\beta_{0W}+\beta_{0S})}
\frac{\partial \phi_\perp}{\partial y}(x, y )}\\
&\boxed{
\Delta p_z = \frac{m_0}{m}\frac{\Delta P_z} {P_0}=0}
\end{align}

\section{Longitudinal profiles}
\subsection{Gaussian profile}

The profile is in the form:
\begin{equation}
\lambda_{0}(z)=\frac{N}{\sqrt{2 \pi} \sigma} e^{-\frac{(z-z_0)^{2}}{2 \sigma^{2}}}
\end{equation}


\subsection{q-Gaussian}

The profile is in the form:
\begin{equation}
\lambda_0(z)=\frac{N\sqrt{\beta}}{C_{q}} e_{q}\left(-\beta (z-z_0)^{2}\right)
\end{equation}
where $e_q$ is the q-exponential function:
\begin{equation}
e_{q}(x)=[1+(1-q) x]_{+}^{\frac{1}{1-q}}
\end{equation}
$C_q$ is a normalization factor dependent on $q$ alone:
\begin{equation}
C_{q}=\frac{\sqrt{\pi} \Gamma\left(\frac{3-q}{2(q-1)}\right)}{\sqrt{q-1} \Gamma\left(\frac{1}{q-1}\right)}
\end{equation}

The parameter beta defines the standard deviation of the distribution:

\begin{equation}
\sigma = \sqrt{\frac{1}{\beta(5-3 q)}} \iff \beta ={\frac{1}{\sigma^2(5-3 q)}}
\end{equation}



These expressions are valid for values of the parameter $q$ is   the range of interest:
\begin{equation}
1<q<\frac{5}{3}
\end{equation}

In general the q-Gaussian is defined outside this range, but for smaller values it has a limited support (not of interest) and for larger values has a not defined standard deviation.

\section{FFT Poisson solver}

\subsection{Notation for Discrete Fourier Transform}
We will use the following notation for the Discrete Fourier Transform of a sequence of length $M$:
\begin{equation}
\hat{a}_k = \text{DFT}_M(a_m) =  \sum_{m=0}^{M-1} a_m\, e^{-j2\pi  \frac{km}{M}}  \quad \text{for } k \in 0, ..., M
\end{equation}
The corresponding inverse transform is defined as:
\begin{equation}
{a}_n = \text{DFT}^{-1}_M(\hat{a}_k) =  \frac{1}{M}\sum_{k=0}^{M-1} \hat{a}_k\, e^{j2\pi  \frac{km}{M}}  \quad \text{for } m \in 0, ..., M
\end{equation}

Multidimensional Discrete Fourier Transforms are obtained by applying sequentially 1D DFTs.. For example, in two dimensions:

\begin{equation}
\begin{split}
\hat{a}_{k_xk_y} &= \text{DFT}_{M_xM_y}\left\{a_{m_xm_y}\right\}  
= \text{DFT}_{M_y} \left\{\text{DFT}_{M_x}\left\{a_{m_xm_y}\right\}\right\}\\  
&=\sum_{m_x=0}^{M_x-1} e^{-j 2\pi  \frac{k_x m_x}{M_x}} 
\sum_{m_y=0}^{M_y-1} e^{-j 2\pi  \frac{k_y m_y}{M_y}} a_{m_xm_y}
\end{split}
\end{equation}
\begin{equation}
\begin{split}
{a}_{n_xn_y} &= \text{DFT}^{-1}_{M_xM_y}\left\{a_{k_x k_y}\right\}  
= \text{DFT}^{-1}_{M_y} \left\{\text{DFT}^{-1}_{M_x}\left\{\hat{a}_{k_x k_y}\right\}\right\}\\  
&=\frac{1}{M_x M_y}\sum_{k_x=0}^{M_x-1} e^{j 2\pi  \frac{k_x m_x}{M_x}} 
\sum_{k_y=0}^{M_y-1} e^{j 2\pi  \frac{k_y m_y}{M_y}} \hat{a}_{k_xk_y}
\end{split}
\end{equation}

\subsection{FFT convolution - 1D case}
The potential can be written as the convolution of a Green function with the charge distribution:
\begin{equation}
\phi(x) = \int_{-\infty}^{+\infty} \rho(x')\,G(x-x') dx'
\label{eq:conv}
\end{equation}

We assume that the source is limited to the region  $[0, L]$:
\begin{equation}
\rho(x) = \rho(x)\,\Pi_{[0,L]}\left(x\right)
\label{eq:rholim}
\end{equation}
where $\Pi_{[a,b]}(x)$ is a rectangular window function defined as:
\begin{equation}
\Pi_{[a,b]}(x) = 
\begin{cases}
1\quad\text{for } x \in [a, b]\\
0\quad\text{elsewhere}
\end{cases}
\end{equation}

We are interested in the electric potential only the region occupied by the sources, so we can compute:
\begin{equation}
\phi_L(x) = \phi(x) \Pi_{[0, L]}\left(\frac{x}{L}\right)
\label{eq:philim}
\end{equation}

We replace Eq.\,\eqref{eq:rholim} and Eq.\,\eqref{eq:philim} into Eq.\eqref{eq:conv}, obtaining:
\begin{equation}
\phi_L(x) = \Pi_{[0,L]}\left( x\right)
\int_{-\infty}^{+\infty} 
\Pi_{[0,L]}\left(x'\right)
\rho(x')\,G(x-x') dx'
\end{equation}
We apply the change of variable $x'' = x - x'$:
\begin{equation}
\phi_L(x) = 
\int_{-\infty}^{+\infty} 
\Pi_{[0,L]}\left({x}\right)
\Pi_{[0,L]}\left({x-x''}\right)
\rho(x-x'')\,G(x'') \,dx''
\label{eq:conv1}
\end{equation}
The integrand vanishes outside the set of the $(x, x'')$ defined by:
\begin{equation}
\begin{cases}
0 < x <{L}\\
0 < (x-x'') <{L}
\end{cases}
\end{equation}

We flip the signs in the second equation, obtaining:
\begin{equation}
\begin{cases}
0 < x <{L}\\
-L < (x''-x) <0
\end{cases}
\end{equation}

Combining the two equations we obtain:
\begin{equation}
-L<-L + x < x'' <x<L
\end{equation}
i.e. the integrand is zero for $-L<x''<L$.
Therefore in Eq.\,\eqref{eq:conv1} we can replace $G(x'')$ with its truncated version:
\begin{equation}
G_{2L}(x'') = G(x'')\,\Pi_{[-L,L]}
\left(
{x''}
\right)
\end{equation}

obtaining:
\begin{equation}
\phi_L(x) = 
\int_{-\infty}^{+\infty} 
\Pi_{[0,L]}\left(\frac{x}{L}\right)
\Pi_{[0,L]}\left(\frac{x-x''}{L}\right)
\rho(x-x'')\,G_{2L}(x'') dx''
\label{eq:conv2}
\end{equation}

Since the two window functions force the integrand to zero outside the region $|x''|<L$ we can replace $G_{2L}(x'')$ with its replicated version:
\begin{equation}
G_{2LR}(x'') = \sum_{n=-\infty}^{+\infty}G_{2L}(x''-2nL) = \sum_{n=-\infty}^{+\infty}G(x'' -2nL)\,\Pi_{[-L,L]}
\left(
\frac{x''-2nL}{2L}
\right)
\label{eq:GLR}
\end{equation}
obtaining:
\begin{equation}
\phi_L(x) = 
\int_{-\infty}^{+\infty} 
\Pi_{[0,L]}\left(\frac{x}{L}\right)
\Pi_{[0,L]}\left(\frac{x-x''}{L}\right)
\rho(x-x'')\,G_{2LR}(x'') dx''
\end{equation}

We can go back to the initial coordinate by substituting $x'' = x-x'$:
\begin{equation}
\phi_L(x) = 
\Pi_{[0,L]}\left(\frac{x}{L}\right)
\int_{-\infty}^{+\infty} 
\rho(x')\,G_{2LR}(x-x') dx'
\end{equation}

This is a cyclic convolution, so we can proceed as follows. We split the integral:
\begin{equation}
\phi_L(x) = 
\Pi_{[0,L]}\left(\frac{x}{L}\right)
\sum_{n=-\infty}^{+\infty}
\int_{2nL}^{2(n+1)L} 
\rho(x')\,G_{2LR}(x-x') \,dx'
\label{eq:conv3}
\end{equation}
In each term we replace $x''' = x'+2nL$:
\begin{equation}
\phi_L(x) = 
\Pi_{[0,L]}\left(\frac{x}{L}\right)
\sum_{n=-\infty}^{+\infty}
\int_{0 }^{2L} 
\rho(x'''-2nL)\,G_{2LR}(x-x'''-2nL) \,dx'''
\label{eq:conv4}
\end{equation}
We use the fact that $G_{2LR}(x)$ is periodic:
\begin{equation}
\begin{split}
\phi_L(x) &= 
\Pi_{[0,L]}\left(\frac{x}{L}\right)
\sum_{n=-\infty}^{+\infty}
\int_{0 }^{2L} 
\rho(x'''-2nL)\,G_{2LR}(x-x''') dx'''\\
\\&=
\Pi_{[0,L]}\left(\frac{x}{L}\right)
\int_{0 }^{2L}  
\sum_{n=-\infty}^{+\infty}
\rho(x'''-2nL)\,G_{2LR}(x-x''') dx'''
\end{split}
\label{eq:conv5}
\end{equation}

We can define a replicated version of $\rho(x)$:
\begin{equation}
\rho_{2LR}(x)= \sum_{n=-\infty}^{+\infty}
\rho(x-2nL)
\end{equation}
noting that this implies:
\begin{equation}
\rho_{2LR}(x)= 0 \quad \text{for } x \in [L, 2L]
\label{eq:zeros}
\end{equation}

We obtain:
\begin{equation}
\phi_L(x) = 
\Pi_{[0,L]}\left(\frac{x}{L}\right)
\int_{0 }^{2L} 
\rho_{2LR}(x')\,G_{2LR}(x-x') dx'
\label{eq:conv6}
\end{equation}

The function:

\begin{equation}
\phi_{2LR}(x) = 
\int_{0 }^{2L} 
\rho_{2LR}(x')\,G_{2LR}(x-x') dx'
\label{eq:confin}
\end{equation}
is periodic of period $2L$. From it the potential of interest can be simply calculated by selecting the first half period $[0, L]$:
\begin{equation}
\phi_L(x) = 
\Pi_{[0,L]}\left(\frac{x}{L}\right)
\phi_{2LR}(x)
\label{eq:sel}
\end{equation}

To compute the convolution in Eq.\,\ref{eq:confin} we expand $\phi_{2LR}(x)$ in Fourier series:
\begin{equation}
\phi_{2LR}(x) = \sum_{k=-\infty}^{+\infty} \tilde{\phi}_k\, e^{j2\pi k \frac{x}{2L}}
\label{eq:phifour}
\end{equation}
where the Fourier coefficients are given by:
\begin{equation}
\tilde{\phi}_k = \frac{1}{2L}\int_0^{2L} \phi_{2LR}(x)\, e^{-j2\pi k \frac{x}{2L}} \, dx
\label{eq:phik}
\end{equation}

We replace Eq.\,\eqref{eq:confin} into Eq.\,\eqref{eq:phik} obtaining:
\begin{equation}
\hat{\phi}_k = \frac{1}{2L}\int_0^{2L} \int_{0 }^{2L} 
\rho_{2LR}(x')\,G_{2LR}(x-x') \, e^{-j2\pi k \frac{x}{2L}} \,  dx'\, dx
\end{equation}

With the change of variable $x'' = x-x'$ we obtain:
\begin{equation}
\tilde{\phi}_k = 
\frac{1}{2L}
\int_0^{2L} 
\rho_{2LR}(x') e^{-j2\pi k \frac{x'}{2L}}dx'\,
\int_{0 }^{2L} 
\,G_{2LR}(x'') e^{-j2\pi k \frac{x''}{2L}}\,  \,  dx''
\end{equation}

where we recognize the Fourier coefficients of $\rho_{2LR}(x)$ and $\,G_{2LR}(x)$:
\begin{align}
\tilde{\rho}_k = \frac{1}{2L}\int_0^{2L} \rho_{2LR}(x)\, e^{-j2\pi k \frac{x}{2L}} \, dx \label{eq:rhok}\\
\tilde{G}_k = \frac{1}{2L}\int_0^{2L} G_{2LR}(x)\, e^{-j2\pi k \frac{x}{2L}} \, dx \label{eq:Gk}
\end{align}
obtaining simply:
\begin{equation}
\hat{\phi}_k = 2L \, \hat{G}_k \, \hat{\rho}_k
\label{eq:freqconv}
\end{equation}

I assume to have the functions $\rho_{2LR}(x)$ and  $G_{2LR}(x)$ sampled (or averaged) with step:
\begin{equation}
h_x = \frac{2L}{M} = \frac{L}{N}
\end{equation}

I can approximate the integrals in Eqs.\,\eqref{eq:rhok} and\,\eqref{eq:Gk} as:
\begin{align}
\tilde{\rho}_k = \frac{1}{M}\sum_{n=0}^{M-1} \rho_{2LR}(x_n)\, e^{-j2\pi  \frac{kn}{M}}  
= \frac{1}{M} \hat{\rho}_k
\label{eq:rhokfft}\\
\tilde{G}_k = \frac{1}{M}\sum_{n=0}^{M-1} G_{2LR}(x_n)\, e^{-j2\pi  \frac{kn}{M}} 
= \frac{1}{M} \hat{G}_k\label{eq:Gkfft}
\end{align}

where we recognize the Discrete Fourier Transforms:
\begin{align}
\hat{\rho}_k = \text{DFT}_M\left\{ \rho_{2LR}(x_n)\right\}\\
\hat{G}_k = \text{DFT}_M\left\{ G_{2LR}(x_n)\right\}
\end{align}



Using Eq.\,\eqref{eq:phifour} we can obtain a sampled version of $\phi(x)$:
\begin{equation}
\phi_{2LR}(x_n) = 
\sum_{n=0}^{M-1}  
\tilde{\phi}_k\, e^{j2\pi \frac{kn}{M}}
\label{eq:phifft}
\end{equation}
where we have assumed that $\phi(x)$ is sufficiently smooth to allow truncating the sum.


Using Eqs.\,\eqref{eq:freqconv}, \eqref{eq:rhokfft} and\,\eqref{eq:Gkfft}  we obtain:
\begin{equation}
\phi_{2LR}(x_n) = 
2L \sum_{n=0}^{M-1}  
\tilde{G}_k \, \tilde{\rho}_k\, e^{j2\pi \frac{kn}{M}} 
= 
\frac{2L}{M^2}
\sum_{n=0}^{M-1}  
\hat{G}_k \, \hat{\rho}_k\, e^{j2\pi \frac{kn}{M}} 
\label{eq:phifftsimpl}
\end{equation}

This can be rewritten as:
\begin{equation}
\phi_{2LR}(x_n) = 
\frac{1}{M}
\sum_{n=0}^{M-1}  
(h_x\hat{G}_k) \, \hat{\rho}_k\, e^{j2\pi \frac{kn}{M}} 
=\text{DFT}_M^{-1}\left\{\phi_k
\right\}
\label{eq:invfft}
\end{equation}
where 
\begin{equation}
\hat{\phi}_k =h_x\hat{G}_k \, \hat{\rho}_k
\label{eq:phiknint}
\end{equation}
We call ``Integrated Green Function'' the quantity:
\begin{equation}
G_{2LR}(x_n) = h_x G_{2LR}(x_n)
\end{equation}
we introduce the corresponding Fourier transform:
\begin{equation}
\hat{G}_k^\text{int} = \text{DFT}_M\left\{ G_{2LR}^\text{int}(x_n)\right\}
\end{equation}
Eq.\,\eqref{eq:phiknint} can be rewritten as:
\begin{equation}
\boxed{
\hat{\phi}_k =\hat{G}_k^\text{int} \, \hat{\rho}_k}
\end{equation}

In summary the potential at the grid nodes can be computed as follows:
\begin{enumerate}
\item We compute the Integrated Green function at the grid points in the range $[0, L]$:
\begin{equation}
G_{2LR}^\text{int}(x_n) = \int_{x_n-\frac{h_x}{2}}^{x_n+\frac{h_x}{2}} G(x) dx
\end{equation}
\item We extend to the interval $[L, 2L]$ using the fact that in this interval:
\begin{equation}
G^\text{int}_{2LR}(x_n) = G^\text{int}_{2LR}(x_n-2L) =  G^\text{int}_{2LR}(2L-x_n)
\end{equation}
where the first equality comes from the periodicity of $G^\text{int}_{2LR}(x)$ and the second from the fact that $G(x)$ is an even function (i.e. $G(x) = G(-x)$).
Note that for $x_n \in [L, 2L]$ we have that $2L-x_n \in [0, L]$ so we can reuse the values computed at the previous step.
\item We transform it:
\begin{equation}
\hat{G}^\text{int}_k = \text{DFT}_{2N}\left\{ G_{2LR}(x_n)\right\}
\end{equation}
\item We assume that we are given $\rho(x_n)$ in the interval $[0, L]$. From this we can obtain $\rho_{2LR}(x_n)$ over the interval $[0, 2L]$ simply extending the sequence with zeros (see Eq.\,\eqref{eq:zeros}).
\item We transform it:
\begin{equation}
\hat{\rho}_k = \text{DFT}_{2N}\left\{ \rho_{2LR}(x_n)\right\}
\end{equation}
\item We compute the potential in the transformed domain:
\begin{equation}
\hat{\phi}_k = \hat{G}^\text{int}_k \hat{\rho}_k \quad \text{for } k\in [0, 2N]
\end{equation}
\item We inverse-transform:
\begin{equation}
\phi_{2LR}(x_n)  = \text{DFT}_{2N}^{-1}\left\{\hat{\phi}_k\right\}
\end{equation}
which provides the physical potential in the range $[0, L]$:
\begin{equation}
\phi(x_n)  = \phi_{2LR}(x_n)  \quad \text{for } x_n\in [0, L]
\end{equation}
\end{enumerate}


\subsection{Extension to multiple dimensionss}

The procedure described above can be extended to multiple dimensions by applying the same reasoning for all coordinates. 
This gives the following procedure:
\begin{enumerate}
\item We compute the Integrated Green function at the grid points in the volume $[0, L_x]\times[0, L_y]\times[0, L_z]$:
\begin{equation}
G_{2LR}^\text{int}(x_{n_x}, y_{n_y}, z_{n_z}) = 
\int_{x_{n_x}-\frac{h_x}{2}}^{x_{n_x}+\frac{h_x}{2}} dx
\int_{y_{n_y}-\frac{h_y}{2}}^{y_{n_y}+\frac{h_y}{2}} dy
\int_{z_{n_z}-\frac{h_z}{2}}^{z_{n_z}+\frac{h_z}{2}} dz\,
G(x,y,z)
\end{equation}
\item We extend to the region $[0, 2L_x]\times[0, 2L_y]\times[0, 2L_z]$ using the fact that:
\begin{multline}
G^\text{int}_{2LR}(x_n, y_n, z_n) = G^\text{int}_{2LR}(x_n-2L_x, y_n, z_n) =  G^\text{int}_{2LR}(2L_x-x_n, y_n, z_n)\\
\text{for } x_n \in [L_x, 2L_x], y_n \in [0, 2L_y], z_n \in [0, 2L_z]
\end{multline}
\begin{multline}
G^\text{int}_{2LR}(x_n, y_n, z_n) = G^\text{int}_{2LR}(x_n, y_n-2L_y, z_n) =  G^\text{int}_{2LR}(x_n, 2L_y-y_n,  z_n)\\
\text{for } y_n \in [L_y, 2L_y], x_n \in [0, 2L_x], z_n \in [0, 2L_z]
\end{multline}
\begin{multline}
G^\text{int}_{2LR}(x_n, y_n, z_n) = G^\text{int}_{2LR}(x_n, y_n, z_n-2L_z) =  G^\text{int}_{2LR}(x_n, y_n,  2L_z-z_n)\\
\text{for } z_n \in [L_z, 2L_z], x_n \in [0, 2L_x], y_n \in [0, 2L_y]
\end{multline}
This allows reusing the values computed at the previous step.
\item We transform it:
\begin{equation}
\hat{G}^\text{int}_{k_x k_y k_z} = 
\text{DFT}_{2N_x 2N_y 2N_z}\left\{ G_{2LR}(x_n, y_n, z_n)\right\}
\end{equation}
\item We assume that we are given $\rho(x_n, y_n, z_n)$ in the region $[0, L_x]\times[0, L_y]\times[0, L_z]$. From this we can obtain $\rho_{2LR}(x_n)$ over the region $[0, 2L_x]\times[0, 2L_y]\times[0, 2L_z]$ simply extending the matrix with zeros (see Eq.\,\eqref{eq:zeros}).
\item We transform it:
\begin{equation}
\hat{\rho}^\text{int}_{k_x k_y k_z} = 
\text{DFT}_{2N_x 2N_y 2N_z}\left\{ \rho_{2LR}(x_n, y_n, z_n)\right\}
\end{equation}
\item We compute the potential in the transformed domain:
\begin{equation}
\hat{\phi}_{k_x k_y k_z} = \hat{G}^\text{int}_{k_x k_y k_z} \, \hat{\rho}_{k_x k_y k_z} \quad \text{for } k_{x/y/z}\in [0, 2N_{x/y/z}]
\end{equation}
\item We inverse-transform:
\begin{equation}
\phi_{2LR}(x_n, y_n, z_n)  = \text{DFT}_{2N_x 2N_y 2N_z}^{-1}
\left\{\hat{\phi}_{k_x k_y k_z}\right\}
\end{equation}
which provides the physical potential in the region $[0, L_x]\times[0, L_y]\times[0, L_z]$:
\begin{multline}
\phi(x_n, y_n, z_n) = \phi_{2LR}(x_n, y_n, z_n)  
\text{ for } (x_n, y_n, z_n) \in [0, L_x]\times[0, L_y]\times[0, L_z]
\end{multline}
\end{enumerate}

 
\subsection{Green functions for 2D and 3D Poisson problems}

\subsubsection{3D Poisson problem, free space boundary conditions}

For the equation:
\begin{equation}
\nabla^2 \phi(x,y,z) = -\frac{1}{\varepsilon_0} \rho(x,y,z)
\end{equation}
where:
\begin{equation}
\nabla = \left(\frac{\partial}{\partial x}, 
                      \frac{\partial}{\partial y},
                      \frac{\partial}{\partial z} \right)
\end{equation}
the solution can be written as 
\begin{equation}
\phi(x, y, z) = \iiint_{-\infty}^{+\infty} \rho(x', y', z')
   \,G(x-x', y-y', z-z')\,dx'\,dy'\,dz'
\end{equation}
where:
\begin{equation}
G(x, y, z) = \frac{1}{4\pi\varepsilon_0}\frac{1}{
\sqrt{x^2 +y^2 +z^2}
}
\end{equation}

The corresponding integrated Green function can be written as:
\begin{align}
G_{2LR}^\text{int}(x_{n_x}, y_{n_y}, z_{n_z}) =& 
\int_{x_{n_x}-\frac{h_x}{2}}^{x_{n_x}+\frac{h_x}{2}} dx
\int_{y_{n_y}-\frac{h_y}{2}}^{y_{n_y}+\frac{h_y}{2}} dy
\int_{z_{n_z}-\frac{h_z}{2}}^{x_{n_z}+\frac{h_z}{2}} dz\,
G(x,y,z)\\
= &+F\left(x_{n_x}+\frac{h_x}{2}, y_{n_x}+\frac{h_y}{2}, z_{n_x}+\frac{h_z}{2}\right) \\
    &- F\left(x_{n_x}+\frac{h_x}{2}, y_{n_x}+\frac{h_y}{2}, z_{n_x}-\frac{h_z}{2}\right)\\
    &- F\left(x_{n_x}+\frac{h_x}{2}, y_{n_x}-\frac{h_y}{2}, z_{n_x}+\frac{h_z}{2}\right)\\ 
    &+ F\left(x_{n_x}+\frac{h_x}{2}, y_{n_x}-\frac{h_y}{2}, z_{n_x}-\frac{h_z}{2}\right)\\
    &-F\left(x_{n_x}-\frac{h_x}{2}, y_{n_x}+\frac{h_y}{2}, z_{n_x}+\frac{h_z}{2}\right) \\
    &+ F\left(x_{n_x}-\frac{h_x}{2}, y_{n_x}+\frac{h_y}{2}, z_{n_x}-\frac{h_z}{2}\right)\\
    &+ F\left(x_{n_x}-\frac{h_x}{2}, y_{n_x}-\frac{h_y}{2}, z_{n_x}+\frac{h_z}{2}\right)\\ 
    & - F\left(x_{n_x}-\frac{h_x}{2}, y_{n_x}-\frac{h_y}{2}, z_{n_x}-\frac{h_z}{2}\right)
\end{align}
where $F(x,y,z)$ is a primitive of $G(x,y,z)$, which can be obtained as:
\begin{equation}
F(x,y,z) = 
\int_{x_0}^{x} dx
\int_{y_0}^{y} dy
\int_{z_0}^{x} dz\,
G(x,y,z)
\end{equation}
with $(x_0, y_0, z_0)$ being an arbitrary starting point.

An expression for $F(x,y,z)$ is the following
\begin{align}
F(x,y,z) =&\frac{1}{4\pi\varepsilon_0}\iiint \frac{1}{\sqrt{x^{2}+y^{2}+z^{2}}} d x d y d z\\ 
= \frac{1}{4\pi \varepsilon_0}&\left[-\frac{z^{2}}{2} \arctan \left(\frac{x y}{z \sqrt{x^{2}+y^{2}+z^{2}}}\right)\right.
-\frac{y^{2}}{2} \arctan \left(\frac{x z}{y \sqrt{x^{2}+y^{2}+z^{2}}}\right)\\
&-\frac{x^{2}}{2} \arctan \left(\frac{y z}{x \sqrt{x^{2}+y^{2}+z^{2}}}\right) 
+y z \ln \left(x+\sqrt{x^{2}+y^{2}+z^{2}}\right)\\
&\left. +x z \ln \left(y+\sqrt{x^{2}+y^{2}+z^{2}}\right)
+x y \ln \left(z+\sqrt{x^{2}+y^{2}+z^{2}}\right)\right]
\end{align}

Note that we need to choose the first cell center to be in (0,0,0) for evaluation of the integrated Green function. Therefore the cell edges have non zero coordinates and the denominators in the formula will always be non-vanishing.

\subsubsection{2D Poisson problem, free space boundary conditions}

For the equation:
\begin{equation}
\nabla_\perp^2 \phi(x,y) = -\frac{1}{\varepsilon_0} \rho(x,y)
\end{equation}
where:
\begin{equation}
\nabla = \left(\frac{\partial}{\partial x}, 
                      \frac{\partial}{\partial y} \right)
\end{equation}
the solution can be written as 
\begin{equation}
\phi(x, y) = \iiint_{-\infty}^{+\infty} \rho(x', y')
   \,G(x-x', y-y')\,dx'\,dy'
\end{equation}
where:
\begin{equation}
G(x, y) = -\frac{1}{4\pi\varepsilon_0} \log\left( \frac{x^2 + y^2}{r_0^2}\right)
\end{equation}
where $r_0$ is arbitrary constant which has no effect on the evaluated fields (changes the potential by an additive constant). 

The corresponding integrated Green function can be written as:
\begin{align}
G_{2LR}^\text{int}(x_{n_x}, y_{n_y}) =& 
\int_{x_{n_x}-\frac{h_x}{2}}^{x_{n_x}+\frac{h_x}{2}} dx
\int_{y_{n_y}-\frac{h_y}{2}}^{y_{n_y}+\frac{h_y}{2}} dy\,
G(x,y,z)\\
= &+F\left(x_{n_x}+\frac{h_x}{2}, y_{n_x}+\frac{h_y}{2}\right)\\
    &-F\left(x_{n_x}+\frac{h_x}{2}, y_{n_x}-\frac{h_y}{2}\right)\\
    &-F\left(x_{n_x}-\frac{h_x}{2}, y_{n_x}+\frac{h_y}{2}\right)\\ 
    &+F\left(x_{n_x}-\frac{h_x}{2}, y_{n_x}-\frac{h_y}{2}\right)\\
\end{align}
where $F(x,y)$ is a primitive of $G(x,y)$, which can be obtained as:
\begin{equation}
F(x,y) = 
\int_{x_0}^{x} dx
\int_{y_0}^{y} dy\,
G(x,y)
\end{equation}
where $(x_0, y_0)$ is an arbitrary starting point.

An expression for $F(x,y,z)$ is the following (where we have chosen $r_0=1$:
\begin{align}
F(x,y,z) &=-\frac{1}{4\pi\varepsilon_0}\iint \ln \left(x^{2}+y^{2}\right) dx/,dy\\
&=\frac{1}{4\pi\varepsilon_0}\left[3 x y-x^{2} \arctan (y / x)-y^{2} \arctan (x / y)-x y \ln \left(x^{2}+y^{2}\right)\right]
\end{align}

Note that we need to choose the first cell center to be in (0,0) for evaluation of the integrated Green function. Therefore the cell edges have non zero coordinates and the denominators in the formula will always be non-vanishing.

\addcontentsline{toc}{section}{References}
\bibliographystyle{unsrt}
\bibliography{bibliography}



\end{document}

